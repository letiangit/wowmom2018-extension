\section{Main idea \label{sec:main_idea}}
Table \ref{tab:variables} lists all the variable used in the algorithm description, figure ~\ref{fig:sumo-throughput} and \ref{fig:sumo-packetNum} give examples of SUMO results. \\

\textbf{Derived from the SUMO:} \\
For stations sending one packet \textbf{every} second
%, with stations number, we can get the \{RAW group duration, slot number\}, which gives \textbf{highest throughput} or \textbf{maximal number of successfully transmitted packets}. It can be represented as:\\
%\{$\mathcal{S}^r$, $\mathcal{D}^r_\textit{opt}$, $\mathcal{SL}^r_\textit{opt}$, $\pi^r_\textit{opt}$\} \\

\begin{equation}
\pi = f(\mathcal{S}^r, \mathcal{D}^r, \mathcal{SL}^r)
\end{equation}

It can be further extend. For stations sending one packet \textbf{every m} second
\begin{equation}
\pi = f(\mathcal{S}^r/m, \mathcal{D}^r, \mathcal{SL}^r)
\end{equation}

m should satisfy:\\
\begin{enumerate}
\item Integer, or
\item $1/m$ is integer, $1 < 1/m <= 5(1000ms/204.8ms) $. \\
Boundary of $1/m$: In SUMO, each station is expected to send at most one packets per beacon interval (one packet per seconds, beacon interval(204.8 ms) is less than one seconds). When $1/m$ is larger than 5 (1000ms/204.8ms), a station has more than one packet to send for one beacon interval, the model derived from SUMO \textbf{may} not work (not very sure). In theory, $SL$ should not be changed. \\

\textbf{Question, is slot needed for SUMO modeling?}



\end{enumerate}







\textbf{The target scenario}: 
%Stations  \textbf {sending one packet per $x$ (integer) second}. maximize number of packets get transmitted successfully (per beacon interval).\\
\begin{enumerate}
\item Station $s$ \textbf {sending one packet per $x=t^s_{int}$ (integer) second}.
\item Transmission interval $X$ changes from station to station.
\item For each $x \in X$, there are $N_x$ stations.
\item $N$ stations in total in the network.
\item Traffic load $T = \sum_{x \in \mathcal{X}}   N_x * Payload /x$.
\end{enumerate} 
% \begin{equation}
%  \sum_{x \in \mathcal{X}}   N_x * Payload /x
% \end{equation}
\textbf{Objective}: \\
Maximize number of packets get transmitted successfully (per beacon interval). \\
\textbf{Steps}:
\begin{itemize}
\item Station assignment for one beacon interval. \\
\textbf{Goal}: traffic load balance among each beacon interval.\\
\textbf{Input}: \\
See the target scenario. \\
\textbf{Optimization}: \\
\begin{equation}
\min std. T^B, B={b_1, b_2, ..., b_m} \\
\end{equation}

Subject to:
\begin{equation}
aver. \sum_{b \in \mathcal{B}}  T^b = T
\end{equation}
Station $s \in S$ is assigned to beacon at every $t^s_{int}$ beacon interval.

\textbf{Output}: \\
For beacon $b \in B$, \{$S^b, BI, T^b$\}, $T^b$ traffic load for beacon interval $b$ \\
% \item Station Cluster, satisfy the SUMO requirement, making each station \textbf{sending one packet every its channel accessing interval}.\\
% Cluster stations $S$ with same $x \in X$, get a set of clusters $K$. For example, cluster $k \in K$ contains station with $x=t^k_{int}$. Furthermore, evenly divide cluster $k \in K$ into $t^k_{int}$ clusters to get a set of cluster \{$k_1$,$k_2$, ..., $k_{t^k_{int}}$\} \textcolor{red}{(same transmission interval)}. Assign these cluster to beacons in a round-robin way, each cluster is allowed to transmitted every $t^k_{int}$ beacon interval. Thus, a set of clusters $K^b$ are assigned to one beacon interval, it includes $|X|$ clusters, each cluster has different transmission interval. 
\item Station grouping, including group duration, slot number. \\
\textbf{Goal}: Maximize number of packets get transmitted successfully (one beacon interval).\\
%With the results of cluster and SUMO, further divide the cluster into RAW groups, maximize the number of packets get transmitted successfully (per beacon interval).
\textbf{Input for beacon interval $b$}: \\
\{$S^b, BI, T^b$\}, $T^b$ traffic load for beacon interval $b$ \\
\textbf{Optimization for beacon interval $b$}: 
\begin{equation}
\max \sum_{r \in \mathcal{R}^b}  f(\mathcal{S}^r, \mathcal{D}^r, \mathcal{SL}^r) \\
\end{equation}

Subject to
\begin{equation}
\sum_{r \in R^b} \mathcal{D}^r \leq BI
\end{equation}

\begin{equation}
\sum_{r \in R^b} \mathcal{S}^r = S^b
\end{equation}

% \begin{equation}
% k_i \cap k_j = \emptyset, \quad k_i, k_j \subset \mathcal{K}^b
% \end{equation}

\begin{equation}
r_i \cap r_j = \emptyset, \quad r_i, r_j \subset \mathcal{R}^b
\end{equation}

\textbf{Output for beacon interval $b$}: \\
$R^b$, for each $r \in R^b$, \{${S}^r, {D}^r, {SL}^r$\}


\end{itemize}


\begin{figure*}[t]
    \subfloat[180 stations \label{fig:traffic-pattern-throughput}]{\includegraphics[width=0.33\textwidth]{image/SUMO_THROUGHPUT/Stations181}}
    \subfloat[240 stations \label{fig:traffic-pattern-throughput}]{\includegraphics[width=0.33\textwidth]{image/SUMO_THROUGHPUT/Stations241}}
    \subfloat[400 stations \label{fig:traffic-pattern-throughput}]{\includegraphics[width=0.33\textwidth]{image/SUMO_THROUGHPUT/Stations401}}
  \caption{Influence of number of RAW group duration and slot number on throughput for different number of stations \label{fig:sumo-throughput}}
\end{figure*}



\begin{figure*}[t]
    \subfloat[180 stations \label{fig:traffic-pattern-throughput}]{\includegraphics[width=0.33\textwidth]{image/SUMO_PACKETSRE/Stations181}}
    \subfloat[240 stations \label{fig:traffic-pattern-throughput}]{\includegraphics[width=0.33\textwidth]{image/SUMO_PACKETSRE/Stations241}}
    \subfloat[400 stations \label{fig:traffic-pattern-throughput}]{\includegraphics[width=0.33\textwidth]{image/SUMO_PACKETSRE/Stations401}}
  \caption{Influence of number of RAW group duration and slot number on successfully transmitted packets for different number of stations 
  \textcolor{red}{"The throughput of one beacon interval represents the successfully transmitted packets in one beacon interval ", "The results of 20 slots seem weird, i will check later"}\label{fig:sumo-packetNum}}
\end{figure*}

\begin{table*}[t]
\centering
\begin{threeparttable}
\caption{Variables and notations introduced in the algorithm description \label{tab:variables}}
\begin{tabular}{lr}
\hline
\textbf{General variables}           & \textbf{Description}  \\
\hline
$\mathcal{BI}$           & Beacon interval \\
$\mathcal{S}$            & Set of all stations\\
\hline
\textbf{Variables  of beacon interval $b$} & \textbf{Description}  \\
\hline
$\mathcal{K}^b$         & Set of clusters in beacon interval $b$ \\
$\pi^b$                 & Number of packet transmissions in beacon interval $b$ \\
\textbf{Variables  of cluster  $k \in \mathcal{K}^b$ } & \textbf{Description}  \\
\hline
$\mathcal{R}^k$         & Set of RAW groups in cluster $k$ \\
$t^k_{int}$   & Transmission interval of cluster $k$\\
\hline
\textbf{Variables  of RAW group $r \in \mathcal{R}^b$ and Cluster  $k \in \mathcal{K}^b$}  & \textbf{Description}  \\
\hline
$\mathcal{S}^r$         & Set of stations assigned to RAW group $r$\\
$\mathcal{D}^r_\textit{opt}$         & Optimal duration of RAW group $r$\\
$\mathcal{SL}^r_\textit{opt}$         & Optimal slot number of RAW group $r$\\
$\pi^r_\textit{opt}$ & Optimal number of packets successfully transmitted of RAW group $r$ \\
$\pi^r$  & Number of packets successfully transmitted of RAW group $r$ \\
%$t^r_\textit{count}$    & RAW slot duration count\\
% \hline
% \textbf{Variables  of station $s \in \mathcal{S}$} & \textbf{Description}  \\
% \hline
% $t^s_\textit{int}$            & Estimated transmission interval$^\star$ \\
% $\pi^s_{b,r}$                 & Number of packets received by the AP in RAW slot $r$ of beacon interval $b$ from station $s$ \\
% $\pi^s_\textit{failed}$                & Number of consecutive failed transmissions of station $S^i$ \\
% $t^s_\textit{next}$                 & Estimated next transmission time$^\star$ \\
% $t^s_\textit{int}$            & Estimated transmission interval$^\star$ \\
% $\hat{t}^s_\textit{int}$            & Real transmission interval$^\star$ \\
% $t^s_\textit{succ}\left[0\right]$	& Last successful transmission time$^\star$ \\
% $t^s_\textit{succ}\left[1\right]$    & Previous to last successful transmission time$^\star$ \\
% $\pi^s_\textit{trans}\left[0\right]$          & Last transmission result, success or failure \\
% $\pi^s_\textit{trans}\left[1\right]$          & Previous to last transmission result, success or failure \\
\hline
\end{tabular}
%  \begin{tablenotes}
%       \small
%       \item[$\star$] Expressed as a multiple of number of beacon intervals.
%     \end{tablenotes}
\end{threeparttable}
\end{table*}


%By comparing figure ~\ref{fig:sumo-throughput} and \ref{fig:sumo-packetNum}, we can find the high throughput in figure~\ref{fig:sumo-throughput} does not always result in large number of successfully transmitted packets.

% * <le.tian@uantwerpen.be> 2017-11-24T09:58:11.070Z:
%
% ^.

Simulation  1:

$N$ stations are estimated to have packets to send, how to optimize? (assume sequential AID requirement can be met.) 

Assume $m$ ($m=N$?) packet are estimated to be send.

\begin{equation}
\pi = f(\mathcal{S}^r, \mathcal{D}^r, \mathcal{SL}^r)
\end{equation}

Optimization, option:\\
1, Use Number of received as an objective. Choose the max $\pi$. \\
2, Use $T$ = $\pi$ / $\mathcal{D}^r$ as an objective. Choose the max $T$.\\
3, use both Number of received and throughput.\\
Choose $T^r > m$, then further choose max $Tr$

4.multiple RAW group.
%Choose max $T^r$, calculate the corresponding $\pi^r$. compare $m$ and $\pi^r$

Objective: $\pi$ and $T$
















% \begin{equation}
% \max \sum_{r \in \mathcal{R}^b}  f_1(\mathcal{S}^{r_1}, \mathcal{D}^{r_1}, \mathcal{SL}^{r_1}) + f_2(\mathcal{S}^{r_1}, \mathcal{D}^{r_1}, \mathcal{SL}^{r_1}) \\
% \end{equation}

\begin{equation}
\min \sum_{r \in \mathcal{R}^b}   \mathcal{D}^{r_1} + \mathcal{D}^{r_2} \\
\end{equation}


Subject to
\begin{equation}
\sum_{r \in R^b} \mathcal{D}^r \leq BI
\end{equation}

\begin{equation}
   f_1(\mathcal{S}^{r_1}, \mathcal{D}^{r_1}, \mathcal{SL}^{r_1}) \geq {T}^{r_1} \\
\end{equation}

\begin{equation}
   f_2(\mathcal{S}^{r_2}, \mathcal{D}^{r_2}, \mathcal{SL}^{r_2}) \geq {T}^{r_2} \\
\end{equation}

or \\



 \begin{equation}
 \max p_1 + p_2 \\
 \end{equation}
 
 
 Subject to
\begin{equation}
\sum_{r \in R^b} \mathcal{D}^r \leq BI
\end{equation}

\begin{equation}
  p_1 = \min \{ f_1(\mathcal{S}^{r_1}, \mathcal{D}^{r_1}, \mathcal{SL}^{r_1}) / {T}^{r_1}, 1 \}  \\
\end{equation}

\begin{equation}
  p_2 = \min \{ f_2(\mathcal{S}^{r_1}, \mathcal{D}^{r_1}, \mathcal{SL}^{r_1}) / {T}^{r_2}, 1  \}  \\
\end{equation}

or \\

 \begin{equation}
 \mathcal {\max} \min \{ \eta_1, \eta_2 \} \\
 \end{equation}
 
Subject to
 
\begin{equation}
\mathcal{S}_1^r \leq S_1
\end{equation}

\begin{equation}
\mathcal{S}_2^r \leq S_2
\end{equation}

\begin{equation}
 f_1(\mathcal{S}^{r_1}, \mathcal{D}^{r_1}, \mathcal{SL}^{r_1}) >= {T}^{S^{r_1}}
\end{equation}

\begin{equation}
 f_2(\mathcal{S}^{r_2}, \mathcal{D}^{r_2}, \mathcal{SL}^{r_2}) >= {T}^{S^{r_2}}
\end{equation}


\begin{equation}
\mathcal{D}^{r_1} + \mathcal{D}^{r_2} \leq BI
\end{equation}

\begin{equation}
\eta_1 = \mathcal{S}_1^r / S_1
\end{equation}

\begin{equation}
\eta_2 = \mathcal{S}_2^r / S_2
\end{equation}























 
