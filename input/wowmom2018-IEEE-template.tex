% \documentclass[conference]{IEEEtran} 
% %\documentclass{beamer} 
% %\documentclass[10pt, conference]{IEEEtran} 


% \usepackage[pdftex]{graphicx}
% %\usepackage[caption=false,font=footnotesize]{subfig}
% %\usepackage{url}
% \usepackage{multirow}
% %\usepackage{subfigure}
% %\usepackage{caption}
% \usepackage[caption=false]{subfig}
% \usepackage{multicol}
% \usepackage{amsmath}
% \usepackage{color}
% \usepackage{balance}
% %\usepackage{hyperref}
% %\setlength{\textfloatsep}{5pt}

% \usepackage{algorithmic}
% \usepackage{pseudocode}
% \usepackage[linesnumbered,vlined,ruled]{algorithm2e}
% \usepackage{threeparttable}
% \usepackage{comment}
% \usepackage{todonotes}
% \usepackage{tabu}
% \usepackage[justification=centering]{caption}
% \usepackage{soul}
% \usepackage{adjustbox}
% \usepackage{commath}
% \usepackage[acronym]{glossaries}
% \usepackage{titlesec}

% %\usepackage[final]{graphicx}



% %\usepackage{mathptmx}
% %\DeclareCaptionFont{xipt}{\fontsize{11}{13}\mdseries}
% %\usepackage[font=xipt,labelfont=bf]{caption}
% %\usepackage[font=times]{caption}
% %\usepackage[font=rm]{caption}

% % \newfontfamily{\ubuntufont}{Times NeW Roman}
% % \DeclareCaptionFont{ubuntu}{\ubuntufont}
% \usepackage[font={footnotesize}]{caption}
% % \captionsetup{font=ubuntu}
% %\usepackage{newtxtext,newtxmath}

% \title{IEEE~802.11ah Restricted Access Window Surrogate Model for Real-Time Station Grouping}

% \author{\IEEEauthorblockN{Le~Tian\IEEEauthorrefmark{1}, Michael~Mehari\IEEEauthorrefmark{2}, Serena~Santi\IEEEauthorrefmark{1}, Steven~Latr\'e\IEEEauthorrefmark{1}\IEEEauthorrefmark{2},\\ Eli~De~Poorter\IEEEauthorrefmark{2}, Jeroen Famaey\IEEEauthorrefmark{1}}
% \IEEEauthorblockA{\IEEEauthorrefmark{1}University of Antwerp - imec, IDLab, Department of Mathematics and Computer Science, Belgium}
% \IEEEauthorblockA{\IEEEauthorrefmark{2}Ghent University - imec, IDLab, Department of Information Technology, Belgium}}

% \IEEEoverridecommandlockouts\IEEEpubid{\makebox[\columnwidth]{978-1-5386-4725-7/18/\$31.00 \copyright 2018 IEEE \hfill} \hspace{\columnsep}\makebox[\columnwidth]{ }} 
% %978-1-5386-4725-7/18/$31.00 ©2018 IEEE

% \begin{document}

% \newacronym{raw}{RAW}{Restricted Access Window}
\newacronym{sumo}{SUMO}{Surrogate Modeling}
\newacronym{ap}{AP}{Access Point}
\newacronym{iot}{IoT}{Internet of Things}
\newacronym[plural=MCS,longplural={modulation and coding schemes}]{mcs}{MCS}{modulation and coding scheme}
\newacronym{rca}{RCA}{rate control algorithm}
\newacronym{moroa}{MoROA}{Model-Based RAW Optimization Algorithm}
\newacronym{taroa}{TAROA}{Traffic-Aware RAW Optimization Algorithm}
\newacronym{etaroa}{E-TAROA}{Enhanced Traffic-Aware RAW Optimization Algorithm}
\newacronym{rps}{RPS}{RAW Parameter Set}
\newacronym{edca}{EDCA/DCF}{Enhanced Distributed Channel Access and Distributed Coordination Function}
\newacronym{mtc}{MTC}{machine-type communication}
\newacronym{rpd}{RPD}{Relative Percent Difference}
\newacronym[plural=AIDs,longplural={association IDs}]{aid}{AID}{association ID}
\newacronym{csma}{CSMA/CA}{carrier-sense multiple access with collision avoidance}
\newacronym{rrse}{RRSE}{root-relative square error}

% \maketitle

% %%%%%%%%%%%%%%%%%%%%%%%%%%%%%%%%%%%%%%%%%%%%%%%%%%%%%%%%%%%%%%%%%%%%%%%%%%%%%%%%
% \begin{abstract}

% % This electronic document is a ÒliveÓ template. The various components of your paper [title, text, heads, etc.] are already defined on the style sheet, as illustrated by the portions given in this document.
% % % 
% The \gls{raw} mechanism proposed by IEEE~802.11ah promises to address one of the major problems of the \gls{iot}: high channel contention in large-scale densely deployed sensor networks. The \gls{raw} feature allows the \gls{ap} to divide stations into different groups, with only the stations in the same group being allowed to access the channel simultaneously. Existing station grouping strategies only support homogeneous scenarios, where all sensor stations have the same fixed data transmission interval, \gls{mcs} and packet size. In this paper, we present two contributions to address this issue. %First, a surrogate model that predicts \gls{raw} performance given specific network conditions and \gls{raw} configuration parameters. It is fast to train and can be solved in real-time. 
% %Second, the \gls{moroa}, which uses the surrogate model to determine the optimal \gls{raw} configuration in real-time, for heterogeneous stations and dynamic traffic. 
% \textcolor{red}{First, surrogate models that predicts \gls{raw} performance (e.g., throughput, energy) given specific network conditions and \gls{raw} configuration parameters. The models are fast to train and can be solved in real-time. 
% Second, the \gls{moroa}, which uses the surrogate models to determine the \gls{raw} configuration in real-time to optimize multiple objectives, for heterogeneous stations and dynamic traffic.} We compare the accuracy of our surrogate model to %\textcolor{red}{ a state of the art analytical model}, as well as 
% simulation results. Performance of \gls{moroa} is compared to existing \gls{raw} optimization algorithms and traditional 802.11 channel access methods.
% %The results shows that the trained surrogate model can accurately predict \gls{raw} performance with a relative error less than 7\% and 10\% for 95\% and 98\% of the \gls{raw} configurations respectively. \gls{moroa} achieves a throughput up to twice as high as traditional 802.11 channel access functions in dense heterogeneous networks.
% \textcolor{red}{
% The results shows that the trained surrogate models can accurately predict \gls{raw} performance with a relative error less than 7\% and 10\% for 90\% and 95\% of the \gls{raw} configurations respectively. \gls{moroa} achieves  65\% higher throughput and 96\% less enregy consumption than traditional 802.11 channel access functions in dense heterogeneous networks.}


% %\textcolor{red}{TODO: 1 sentence giving accuracy of the model and performance improvement of the algorithm}

% %In this paper, we propose a model-based RAW optimization algorithm \gls{moroa}  to adapt the RAW parameters in real time for large scale heterogeneous sensor network, in which stations can have different MCSs and packet size based on the service they are providing.  It first utilizes the \gls{sumo} to build a RAW model, which can accurately predict the throughput with given RAW parameters (i.e., RAW duration, RAW slot number and station number assigned to the RAW group). Then it estimates the traffic pattern of each stations by using our previously proposed algorithm \gls{etaroa}. Finally, with the estimated traffic in the network and the built RAW model, we optimize the RAW parameters to achieve the performance required by the scenarios, i.e., throughput or fairness among different types of IoT service. By compare to the simulation results, we verified the built RAW model is quite accurate, and can well represent the behavior of the RAW. The simulation results show that, compare to EDCA/DCF and homogeneous RAW optimization algorithms \gls{taroa}, the MoRoA can highly improve the performance of IEEE 802.11ah networks under heterogeneous traffic in real time. This paper contributes with a practical approach to optimizing RAW grouping for large scale heterogeneous sensor network, which is a major leap towards applying RAW mechanism in real real-life sensor networks.


% \end{abstract}


% %%%%%%%%%%%%%%%%%%%%%%%%%%%%%%%%%%%%%%%%%%%%%%%%%%%%%%%%%%%%%%%%%%%%%%%%%%%%%%%%
% \glsresetall
% \section{Introduction}

The recently released long-range and low-power Wi-Fi standard IEEE~802.11ah proposes a novel channel access method, referred to as \gls{raw}. It is a flexible hybrid method, highly suited to provide scalable connectivity to both sparsely and densely deployed low-power devices. \gls{raw} is based on station grouping and attempts to reduce contention and collisions in highly dense deployments by dividing stations into groups and allowing channel access to one group at a time. Consequently, IEEE~802.11ah allows up to 8192 stations to connect to a single \gls{ap}. 

Figure~\ref{fig:RAW} schematically depicts how \gls{raw} works. Specifically, the channel airtime is split into several intervals, some of which are assigned to RAW groups, while others are shared and can be accessed by all stations using the traditional 802.11 \gls{edca}, which rely on \gls{csma} channel access. At fixed intervals a beacon frame is transmitted, carrying a \gls{rps} information element. The \gls{rps} specifies the stations belonging to each group using the start and end \gls{aid}, the group start time, and duration. Moreover, each RAW group consists of one or more equal-duration slots, among which the stations assigned to the RAW group are evenly split (using round robin assignment). The \gls{rps} information element also contains the number of slots, slot format and slot duration count sub-fields, which jointly determine the RAW slot duration.
\textcolor{black}{
Moreover, \gls{raw} supports two types of transmissions indicated by the \gls{csb} sub-field of the \gls{rps} information element. When the \gls{csb} sub-field is set to false, stations should not start a transmission if the remaining time in the current \gls{raw} slot is not enough to complete it. This remaining time is termed as holding period. Alternatively, stations are allowed to continue their ongoing transmissions even after the end of the current RAW slot when the \gls{csb} sub-field is set to true.}
For a more in-depth description of \gls{raw}, the reader is referred to existing literature~\cite{Khorov2015a,Sensor2017}.

\begin{figure}[t]
  \centering
  \includegraphics[width=0.8\columnwidth]{image/raw.pdf}
  \caption{Schematic representation of the \gls{raw} mechanism.\label{fig:RAW}}
\end{figure}

The 802.11ah standard, however, does not specify how to configure the actual \gls{raw} grouping parameters. Additionally, previous research has shown that the optimal \gls{raw} configuration depends on a variety of network-related parameters, such as the number of stations, traffic patterns, and network load~\cite{WoWMoM2016}. Incorrect configuration severely impacts throughput, latency and energy efficiency. As such, there is a need for \gls{raw} optimization algorithms that collect network-related information, and at the start of each beacon interval adapt the \gls{raw} configuration based on the current network conditions. Such an algorithm should be able to calculate a new \gls{raw} configuration in real-time (i.e., at most a few milliseconds), as it needs to use network-related information obtained from the previous beacon interval and calculate a solution before the new \gls{rps} information element is broadcast. Moreover, in order to select the optimal \gls{raw} parameters, it should be able to predict \gls{raw} performance for a given set of parameters under specific network and traffic conditions. This is achieved using some sort \textcolor{red}{of} model of the environment, which takes as input network conditions and a \gls{raw} configuration, and generates as output one or more performance metrics (e.g., throughput or energy consumption).

In the past, several analytic models have been proposed to predict \gls{raw} performance~\cite{Khorov2015b,Wang2015}. However, \st{ such models are too computationally hard to be used in real-time,}  \textcolor{red}{such models rely on simplifications and unrealistic assumptions (e.g., no capture effect, no hidden nodes, homogeneous stations, saturated or static traffic).} As a first contribution, we present an alternative solution to \gls{raw} performance modeling, based on surrogate modeling and trained using realistic simulation results. A surrogate model is based on supervised learning (e.g., Kriging, or neural networks), but can be accurately trained with very few labeled sample data points~\cite{SUMOtoolbox2010} . This is important, as a \gls{raw} configuration depends on many input variables that can take a wide range of values. \st{Moreover, once trained, evaluating the model is equivalent to a constant-time table look-up, which can be easily executed in real-time.}
As there is no IEEE 802.11ah hardware available, training is done using the realistic IEEE~802.11ah ns-3 model~\cite{WNS32016}. It implements the capture effect, hidden stations, as well as heterogeneous stations and  traffic loads.
%By using realistic simulation results, obtained from the IEEE~802.11ah implementation in ns-3~\cite{WNS32016}, the model takes into account the capture effect, as well as heterogeneous stations and different traffic loads.

%As a second contribution, we propose the \gls{moroa}. It clusters stations into groups based on their traffic characteristics and determines the optimal \gls{raw} configuration by solving a non-linear constrained optimization problem. In this problem, the trained surrogate model is used to maximize throughput and fairness in terms of packets delivery ratio. 
As a second contribution, we propose the \gls{moroa}. It clusters stations into groups based on their traffic characteristics and determines the optimal \gls{raw} configuration by solving a non-linear constrained optimization problem. The trained surrogate model is used to maximize throughput and energy efficiency simultaneously. In contrast to existing algorithms, \gls{moroa} supports multi-objective optimization of both throughput and energy, dynamic and heterogeneous traffic. \textcolor{red}{Moreover, by assigning stations into homogeneous groups, \gls{moroa} supports heterogeneous stations with different \glspl{mcs} and packet sizes~\cite{Sensor2017,Sensys2017}.} \st{as well as heterogeneous stations with different MCSs and packet sizes}.

The remainder of this paper is structured as follows. Section~\ref{sec:related_work} surveys related work in terms of \gls{raw} performance modeling and optimization algorithms and compares them to our contributions. Section~\ref{sec:SUMO} details the methodology used to define and train the surrogate model. \gls{moroa} is described in Section~\ref{sec:algorithm}. Section~\ref{sec:evaluation} evaluates the accuracy of our presented model, comparing it to %\textcolor{black}{state of the art analytical models and} 
simulation results. Moreover, performance of \gls{moroa} is evaluated and compared to state of the art \gls{raw} optimization algorithms, as well as the traditional \gls{edca} function of IEEE~802.11. Finally, Section~\ref{sec:conclusion} offers conclusions and a short overview of future work.


%Several algorithms have been proposed to determine suitable RAW parameters. For sensor network traffic with either 1 packet per station or under saturation, some analytical models were proposed. These models are based on different techniques, such as probability theory~\cite{Wang2015,Raeesi2014a}, Markov chains~\cite{Khorov2015b,Zheng2014}, multi-objective game theory~\cite{Bel2014}, and maximum likelihood estimation~\cite{Park2014b}. However, these models are computationally hard, which makes it infeasible to execute them in real-time on actual AP hardware. As an alternative that is computationally feasible and deployable, several partitioning algorithms were proposed. They partition the stations into different RAW slots based on different metrics, such as arbitration inter-frame space number (AIFSN) value~\cite{Ogawa2013}, and station traffic load ~\cite{Chang2015}. However, this information is not known to the AP in reality, also making them infeasible to implement. Recently, we proposed a real-time station grouping algorithm, named TAROA, by estimating the traffic conditions of each station with information only available at the AP~\cite{Sensor2017}. In contrast to other state-of-the-art algorithms it is capable of adjusting its RAW configuration in real-time, in face of station and traffic dynamics. 
% \section{Related Work \label{sec:related_work}}

Since the \gls{raw} feature was proposed, several studies have been conducted on the evaluation of \gls{raw} performance. Raeesi \textit{et al.} demonstrate that the \gls{raw} mechanism can provide substantial improvements in terms of throughput, delay and energy consumption, in particular in highly-loaded dense network scenarios \cite{Raeesi2014a}. In our own previous work~\cite{WoWMoM2016}, we further evaluated the optimal \gls{raw} station grouping configuration under a variety of traffic conditions, such as traffic load, number of stations and \gls{raw} group duration on the optimal number of \gls{raw} groups. These works prove the strong correlation between network and traffic conditions on one hand, and the optimal \gls{raw} configuration on the other. This supports the hypothesis that there is a need for real-time \gls{raw} parameter optimization. 

%\subsection{\gls{raw} performance models}

To determine the optimal \gls{raw} parameters, several analytical models have been proposed to calculate \gls{raw} performance under specific network and traffic conditions. These models make use of different techniques, such as probability theory~\cite{Wang2015}, Markov chains~\cite{Khorov2015b,Zheng2014}, and maximum likelihood estimation~\cite{Park2014b}. Early works assume the network is operating under saturated state, where each station always has packets to send \cite{Zheng2014, Park2014b}. This is not a very realistic assumption for \gls{iot} and \gls{mtc}~\cite{Khorov2015b}. The model proposed by Zheng \textit{et al.} considers both cross and non-cross slot boundary traffic, and is able to calculate the throughput with any given number of stations and \gls{raw} duration~\cite{Zheng2014}. \textcolor{red}{A more accurate mathematical model was recently developed by Lyakhov \textit{et al.} \cite{Evgeny2018}, by taking into account the non-steady state of the back-off function at the beginning of the \gls{raw} period.} Park \textit{et al.} determine the \gls{raw} group duration for a certain number of stations to get maximized successful transmission probability~\cite{Park2014b}. In contrast, more recent works assume each station sends one packet per \gls{raw} slot interval~\cite{Khorov2015b, Wang2015, Bel2014}. Khorov \textit{et al.} built a model to analyze the successful packet transmission probability under a given \gls{raw} group duration~\cite{Khorov2015b}. The model of Wang \textit{et al.} focuses on energy consumption~\cite{Wang2015}. \textcolor{red}{Chang \textit{et al.} took a step further, supporting more diverse traffic demands \cite{Chang2018}. They use the results of two extreme cases (i.e.,  infinite traffic and one packet sent per \gls{raw} slot) to extrapolate a regression-based analytical model that can accurately fit the contention success probability of any traffic patterns. However, the regression model does not take the finite length of the RAW slot into account.}

All existing analytical models share two main shortcomings. First, they are computationally hard. This makes it unfeasible to execute them in real-time on actual \gls{ap} hardware, where at most a few milliseconds are available at the start of the beacon interval to calculate a new \gls{raw} configuration. More importantly, they assume ideal channel conditions, without communication errors, delays or capture effects, \textcolor{red}{ or did not takes into account the important peculiarity of the RAW mechanism.}
The combination of these factors make such models useful only from a theoretical point of view, to analyze the effectiveness of \gls{raw} under a variety of conditions. However, they cannot be used for real-time station grouping under dynamic and realistic traffic conditions. Our proposed surrogate modeling approach aims to address both of these issues.

%\subsection{\gls{raw} optimization algorithms}

In addition to modeling \gls{raw} performance, it is necessary to use this information in real-time, in order to optimize \gls{raw} parameters in an actual network. Current solutions are mainly based on set partitioning. These \gls{raw} optimization algorithms assume the number of \gls{raw} slots and groups is predetermined, and decide how to partition the associated stations among them, according to some metric. Their simplicity makes it computationally feasible to deploy them in real networks. Several algorithms utilize \gls{raw} to mitigate hidden node collisions by splitting mutually hidden nodes into orthogonal groups~\cite{Yoon2016,Damayanti2016}%{Yoon2016,Dong2016,Damayanti2016}
. Chang \textit{et al.} proposed a set partitioning algorithm that assumes the (static) traffic demand of each station is known by the \gls{ap} and load balances them across groups~\cite{Chang2015}. Other existing algorithms focus on simple partitioning metrics, such as fully random~\cite{Ogawa2013} or based on the back-off timer value~\cite{Qutab-Ud-Din2015}, which in reality is not known to the \gls{ap}. Such set partitioning algorithms have several shortcomings. First, high channel contention exists in dense sensor network even without the presence of hidden nodes. Reducing hidden nodes can mitigate collisions to some extent, but is not sufficient. Second, they expect all information, such as the exact traffic intensity of each station, to be readily available at the \gls{ap} side, which in reality is not the case. Third, they assume that the number of groups and slots as well as their duration are predefined, and only the partitioning of stations among them needs to be solved. The number of groups and their duration, however, significantly influence \gls{raw} optimality~\cite{WoWMoM2016}. Finally, none of the presented algorithms take into account traffic dynamics. In a real network, the upstream traffic intensity of stations may change over time for a variety of reasons, and the algorithm should therefore adapt to these changes.

% Recently, we proposed the \gls{taroa} \cite{Sensor2017,Sensys2017}. It adapts the optimal \gls{raw} parameters in real-time by estimating the current traffic conditions, based solely on information available at the \gls{ap}. However, it still has two shortcomings that can be addressed. 
% \textcolor{red}{First, it derives the optimal number of stations to assign to a group based on saturated state simulation results on throughput.} Second, it only supports homogeneous stations (i.e., all stations use the same MCS and average packet size). 

\textcolor{red}{
We had proposed the \gls{taroa} \cite{Sensor2017,Sensys2017}. It adapts the optimal \gls{raw} parameters in real-time by estimating the current traffic conditions, based solely on information available at the \gls{ap}. However, it derives the optimal number of stations to assign to a group based on saturated state simulation results on throughput. Recently, \cite{Tian2018} was proposed. Different from \cite{Sensor2017,Sensys2017},  it uses generic and flexible surrogate models on throughput to determine the optimal \gls{raw} configurations, resulting in significant performance improvements. Moreover, it supports a wide range of traffic conditions and heterogeneous stations. However, \cite{Tian2018} focus on throughput. Energy efficiency is also a major concern in \gls{iot} networks, as sensors are battery powered and they are supposed to work for years. Therefore, in this paper we present an improved version, called \textcolor{yellow}{xx}, it generates surrogate models on both throughput and energy consumption, and use RAW configuration by optimizing both throughput and energy simultaneously. }
%This results in significant performance improvements, especially under non-saturated conditions, which are prevalent in \gls{iot} and \gls{mtc} scenarios. 






%use the simulation results (since the analytic model \cite{Zheng2014} does not support capture effect) of saturated state to derive the optimal number of stations $\sigma^r_\textit{opt}$, while in most case network is not saturated. Second, it can only support homogeneous station, i.e., all station have the same MCS and packet size.




% In contrast to other state-of-the-art algorithms it is capable of adjusting its \gls{raw} configuration in real-time, in face of station and traffic dynamics. Furthermore, by exploiting the “more data” header field \textcolor{white}{and cross slot boundary} feature, a more accurate traffic estimation technique for IEEE 802.11ah sensor stations was proposed, which is integrated into an enhanced version of the Traffic-Adaptive \gls{raw} Optimization Algorithm, referred to as E-TAROA \cite{Sensys2017}. With more accurate traffic estimation in very dense networks with thousands of sensor stations, E-TAROA results in a significantly more optimal \gls{raw} configuration. Specifically, E-TAROA converges significantly faster and achieves up to 23\% higher throughput and 77\% lower latency than the original TAROA algorithm under high traffic loads.



% %\section{IEEE 802.11ah Restricted Access Window \label{sec:80211ah}}


% \section{Surrogate Model of \gls{raw} performance \label{sec:SUMO}}

This sections introduces the surrogate modeling approach and toolbox, as well as its integration with the ns-3 network simulator. Subsequently, we describe how surrogate modeling can be used to train a performance model using supervised machine learning, for estimating throughput of the IEEE~802.11ah \gls{raw} under a wide range of network and traffic conditions.
%Table \ref{tab:variables} provides an overview of the variables used in the description of the RAW modeling and proposed algorithm in section \ref{sec:algorithm}.


\subsection{Surrogate model training methodology}
In order for the \gls{ap} to determine the optimal \gls{raw} parameters in real-time, a \gls{raw} performance model is needed. Given the current network conditions (e.g., network topology, traffic) and a set of \gls{raw} parameter values (e.g., number of groups and slots, group duration, station assignment) the model should estimate performance (e.g., in terms of throughput \textcolor{red}{and energy}). Based on this model, optimal \gls{raw} parameters can subsequently be derived \textcolor{red}{for different objectives} based on current network conditions. However, existing analytical models are computationally expensive and unrealistic due to their assumptions. Surrogate modeling provides the answer~\cite{SUMOtoolbox2010}. A surrogate model is trained at design time, using a limited number of input-output sample data points obtained through simulation or real-life experiments. Surrogate modeling is especially suited for tasks with a large input space, as an accurate model can be trained based on relatively little input data points. Moreover, evaluating the model at runtime is computationally efficient, equivalent to a constant-time table lookup.
%, \textcolor{red}{or can be easily executed by applying fast heuristic optimization algorithms (e.g., particle swarm optimization)} .
This makes surrogate modeling highly suitable for \gls{raw} performance modeling, as the input space is very large, and efficient runtime model evaluation is needed for real-time \gls{raw} parameter selection. Additionally, by using realistic simulation results, a surrogate model does not suffer from the same restrictive assumptions as existing analytical models.



\begin{figure}[t]
  \centering
   \subfloat[General SUMO modeling architecture. \label{fig:sumo}]{\includegraphics[width=0.7\columnwidth]{image/SUMO.pdf}}\\
   \subfloat[Integrated \gls{raw} model training using SUMO and ns-3. \label{fig:SUMO_NS3}]{\includegraphics[width=0.7\columnwidth]{image/sumons3.pdf}}
  \caption{Training methodology used to model \gls{raw} performance.}
\end{figure}

The Matlab \glsreset{sumo}\gls{sumo} Toolbox is a flexible framework for accurate global surrogate modeling~\cite{SUMOtoolbox2010}. %It has been is applicable to a wide range of domains, including automotive, microwave filter, wireless network \cite{mehari2015}\cite{mehari2016}, etc. It supports a large set of plugins, each component is configurable and can easily be added, removed or replaced by custom implementations. The SUMO toolbox can be applied in an black-box fashion,
The general \gls{sumo} modeling architecture is illustrated in Figure~\ref{fig:sumo}. The controller plays the key role, managing the modeling process. First, the user offers a set of initial sample data points (including input and output) to the controller. The controller uses those points to construct an initial surrogate model. Next, with the constructed model, the controller predicts the next input space element from which the expected accuracy improvement is the largest. The modeling process keeps iterating, and terminates once certain stopping conditions are met (e.g., the maximum training time is exceeded).

To train the \gls{raw} model, we used our previously developed IEEE~802.11ah ns-3 simulation module~\cite{WNS32016}. Figure~\ref{fig:SUMO_NS3} shows our adapted training methodology to allow the integration between ns-3 and the \gls{sumo} toolbox.
The modified controller conducts similar tasks as the original \gls{sumo} controller. However, it now directly interfaces with ns-3. When a new sample data point is generated for which the output is unknown, the controller will initiate an ns-3 simulation to determine the output associated with the input parameter values of the data point. The \gls{sumo} toolbox executes the following steps to train the \gls{raw} performance surrogate model, using the same numbering as the arrows in Figure~\ref{fig:SUMO_NS3}:
\begin{enumerate}
\item \label{sm_tr_1} The controller reads the settings of the 802.11ah \gls{raw} experiment, including the general parameters of 802.11ah (cf. Table~\ref{tab:ns3 parameters}) and the input space parameters (cf. Table~\ref{tab:sumo parameters}). 

\item \label{sm_tr_2} The controller generates an initial sample set based on the input space, and starts ns-3 experiments with the required settings. 
%The sample method used is Latin Hypercube Sampling (LHS), which is a stratified sampling method that selects sample points evenly along the input space while ensuring proportional representation of design variables. 

\item \label{sm_tr_3} At the end of each experiment, the controller retrieves the evaluation criterion (e.g., throughput\textcolor{red}{, energy consumption}) of the experiment and builds the sample data space.  

\item \label{sm_tr_4} After the experiments with the initial sample data set, the controller builds the surrogate model and calculates the cross validation score. %If the score is below a threshold, the experiment stops executing. Otherwise, it continues with the next step.

\item \label{sm_tr_5} \textcolor{red}{The sampling strategy is applied to select the next sample data points to improve the model accuracy.} 

%The built surrogate model estimates the next sample data point to evaluate with the highest expected accuracy improvement. 

\item \label{sm_tr_6} The controller starts the next ns-3 experiments for the newly selected sample data points. 

\item \label{sm_tr_7} The controller reads the output of the experiment and updates the sample data space, then goes back to step \ref{sm_tr_4}. This process continues until the stop conditions are met.
%cross validation score satisfies the threshold, or the maximum allowed training time has been reached.
\end{enumerate}


In our experiments, the SUMO toolbox is configured to use the latin hypercube sampling method~\cite{FAViana2013} to generate 100 initial sample data points, Kriging interpolation is used to train the model~\cite{JMLRv15couckuyt14a},  FLOLA-Voronoi sampling for generating the next sample points~\cite{van2015fuzzy}, and the 10-fold cross-validation with a \gls{rrse} measure to evaluate the model accuracy~\cite{van2016sensitivity}. The training stops once the cross-validation score lower than or equal to  $0.10$ \todo{change the number} (2 digits of precision) occurs $10$ times in succession, or the number of training data points exceeds $2500$. 

%\textcolor{red}{In k-fold cross-validation, the original sample is randomly partitioned into k equal size subsamples. Of the k subsamples, a single subsample is retained as the validation data for testing the model, and the remaining k-1 subsamples are used as training data. The cross-validation process is then repeated k times (the folds), with each of the k subsamples used exactly once as the validation data. The k results from the folds can then be averaged (or otherwise combined) to produce a single estimation}

\subsection{Training scenario}

In this section, we provide an overview of the static simulation environment parameters used during training. Since the goal of \gls{raw} is scalability under uplink traffic, we consider an \gls{iot} sensing scenario, where sensors periodically monitor the environment and send the resulting data to a server (via the \gls{ap}). The PHY and MAC layer parameters are shown in Table~\ref{tab:ns3 parameters}. Given the low-power nature of battery powered sensors, the PHY layer parameters are configured based on the low-power 802.11ah radio hardware prototype developed by Ba et al.~\cite{Ba2016}, with a transmission power of 0~dBm, a gain of 0~dBi (for both sensor and \gls{ap}), and noise figure of 6.8~dB. In order to obtain a model that is independent of the actual deployment of stations, stations are randomly placed around the AP within a maximum radius. The size of the stations' transmit queues is configured to be 10 packets. \textcolor{red}{As the \gls{raw} duration could be shorter than beacon interval, there are still some remaining shared airtime outside of the \gls{raw} group. Contrast to the IEEE~802.11ah specification, in order to purely evaluate the \gls{raw} performance,  we disable packets transmission on  the shared airtime during the training. As no off-the-shelf IEEE 802.11ah hardware is available, we taken values of energy consumption of each state (i.e., i.e., transmitting, receiving, idle and sleeping) from the IEEE~802.11ah transceiver developed by Ba et al. \cite{ba20174mw}, and integrate them into the IEEE~802.11ah ns-3 simulator to track the energy consumption. %values of the transceiver developed by Ba et al.[23]. The values used to calculate the energy consumptionare 4.4 mW for the reception and idling and 7.2 mW forthe transmitting.
} As the RAW optimization algorithm proposed in Section~\ref{sec:algorithm} groups together stations that use the same \gls{mcs}, the model assumes a fixed \gls{mcs}. However, as \gls{raw} performance depends on the \gls{mcs} used, a different model is to be trained for each \gls{mcs} that stations are expected to use. We illustrate this by developing a separate a high-throughput (HT) and low-throughput (LT) model for two different \gls{mcs} parameter sets. This can be trivially extended to other \gls{mcs} values.  For training simplicity, we assume each station sends one packet per second. However, we show in Section~\ref{sec:algorithm} how this model can be used to calculate \gls{raw} performance under arbitrary data transmission intervals. Each experiment runs for 60 seconds of simulated time. As \gls{raw} is configured in each beacon interval of 204.8~ms, the results of every simulated configuration are averaged over 290 beacon intervals, ensuring the generality of the trained model.

%minimizing variety  the  the  high readability on the results  

%\textcolor{red}{TODO: we need to mention how many times (i.e., beacon intervals) each experiment is repeated for training, and show that this makes it robust to the random effects of CSMA...}

\begin{table}[t]
\centering
\renewcommand{\arraystretch}{1.2}
%\tiny
\caption{\textsc{Simulation parameters used during training}\label{tab:ns3 parameters}}
\begin{tabular}{ll}
\hline
\textbf{PHY parameters}             & \textbf{Value}  \\
\hline
%Frequency (Mhz)                      & 868 \\
TX power (dBm)             & 0    \\
TX/RX gain (dB)                        & 0     \\
Noise Figure (dB)               & 6.8      \\  
%Coding method                  & BCC \\
Propagation model         & Outdoor \\ %, macro~\cite{Hazmi2012} \\
Error rate model               & YansErrorRate \\
\hline
\textbf{MAC parameters}               & \textbf{Value}  \\
\hline
% CWmin                          & 15 \\
% CWmax                          & 1023      \\
%Duration of AIFS (us)\textcolor{red}{change to DIFS}            & 316      \\
Traffic access categories             & \textcolor{black}{ AC\_{BE} } \\
Duration of AIFS ($\mu$s)             & 264      \\
Duration of SIFS ($\mu$s)            & 160      \\
%RTS/CTS                        & not enabled  \\
Beacon interval (ms)           & 204.8 \\
%Cross slot boundary            & Enabled \\
%Rate control algorithm         & constant  \\
Size of transmission queue (packets)  & 10  \\
Packet transmission interval (s)        & 1  \\
Station distribution           & random  \\
\hline
\textbf{Low-Throughput (LT) parameters}  & \textbf{value} \\
\hline
Wi-Fi mode                     & MCS1, 1 Mhz \\
Average payload size (bytes)           & 64   \\
Topology radius (m)     & 200  \\
\hline
\textbf{High-Throughput {HT} parameters}  & \textbf{value}  \\
\hline
Wi-Fi mode                     & MCS9, 1 Mhz \\
Average payload size (bytes)           & 256   \\
Topology radius (m)     & 80  \\
\hline
\end{tabular}
\end{table}

\begin{comment}
\begin{table*}[t]
\centering
\begin{threeparttable}
\caption{Variables and notations introduced in the SUMO modeling and algorithm description \label{tab:variables}}
\begin{tabular}{lr}
\hline
\textbf{General variables}           & \textbf{Description}  \\
\hline
$\mathcal{B}$           & Beacon interval \\
$\mathcal{S}$            & Set of all stations\\
\hline
\textbf{Variables  of RAW group $r$}   & \textbf{Description}  \\
\hline
${s}_r$         & Number of stations assigned to group $r$ \\
${d}_r$         & Duration of group $r$ \\
${l}_r$         & RAW slot number of group $r$ \\
\hline
\end{tabular}
%  \begin{tablenotes}
%       \small
%       \item[$\star$] Expressed as a multiple of number of beacon intervals.
%     \end{tablenotes}
\end{threeparttable}
\end{table*}
\end{comment}


\subsection{Input and output parameters for RAW modeling}

% In order to build the surrogate model, a representative set of data samples must be collected to build a model with sufficient accuracy.
% Data samples are collected in such a way that the model
% accuracy can be maximized while minimizing the number
% of experiments needed. In sequential steps, a well-chosen set of
% experiments are performed by making a balanced trade-off
% between two different criteria, namely exploration and
% exploitation

The surrogate model aims to accurately predict \textcolor{red}{performance (e.g., throughput, energy consumption}, for a RAW group $r$ with duration $d_r$, consisting of $s_r$ slots, and with $n_r$ stations assigned to it. The resulting model can be represented as functions $\mathcal{F}$ , as follows:
\begin{equation} \label{eq:sumo_model}
t_r = \mathcal{F}(n_r, d_r, s_r) 
\end{equation}
In addition to throughput \textcolor{red}{and energy consumption}, the simulator calculates a variety of other performance metrics, such as packet loss and latency. As such, the same methodology as described here can be used to train a model for predicting these other performance metrics. However, to simplify the explanation, we focus on throughput \textcolor{red}{and energy}.

\subsubsection{Input parameters design}
To build the \gls{sumo} model, the input parameter space needs to be defined. It consists of the minimum and maximum value of each parameter, as well as a step size. The minimum and maximum can be determined based on expert knowledge of \gls{raw} performance, as well as legal values defined by the IEEE~802.11ah standard. The range of the number of stations $n_r$ in a \gls{raw} group should span from low to high traffic conditions, so the trained model gives accurate results independent of the density. From our previous studies on \gls{raw} performance~\cite{WoWMoM2016}, and based on the parameters listed in Table~\ref{tab:ns3 parameters}, a minimum value of $60$ and maximum value of $400$ stations per group were deemed to cover all possible traffic conditions. The RAW group duration should be large enough to send at least 1 packet successfully, and at most equal to the duration of the beacon intervals. As such, $d_r$ is varied between $40960$~$\mu$s and $204800$~$\mu$s. The number of slots $s_r$ is bound between $1$ and $64$, as per the IEEE standard. However, a very high number of slots leaves not enough time within a slot to successfully transmit a packet. As such, we limit $s_r$ between 1 and 50.

The actual step size of $n_r$ and $s_r$ is $1$, as they are integer variables. The parameter $d_r$ has a step size of $120$~$\mu$s, as defined in the 802.11ah standard. This results in a total input space of $2.3 \times 10^7$ possible data points. This is too high to properly train the model in a feasible amount of time. To alleviate this, we experimentally determined a good step size for each of the three parameters, leveraging the trade-off between accuracy and training speed. For $n_r$ and $s_r$ a step size of 5 was selected. It was found that the \gls{raw} duration $d_r$ has a high sensitivity. As such, a small value of $5120$~$\mu$s was chosen as its step size. This results in a significantly reduced input space of $25047$ data points. Table~\ref{tab:sumo parameters} summarizes the selected input parameter space. Note that a slot count $s_r$ equal to $0$ is not legal, and $s_r$ therefore takes on values from the set $\left\{1, 5, 10, ..., 50\right\}$. Results for data points outside the reduced input parameter space are obtained via linear interpolation of the two nearest data points included in the model.
%\textcolor{red} {The obtained surrogate model is extended by using linear interpolation to predict results of RAW configurations that are outside the reduced input parameters space.}

%\textcolor{red}{TODO: we need to explain how you can obtain results for values that are outside of the listed training values (e.g., $n_i = 7$). That is currently not mentioned at all.}

\begin{table}[t]
\centering
\renewcommand{\arraystretch}{1.2}
%\tiny
\caption{\textsc{Definition of the input parameter space}\label{tab:sumo parameters}}
\begin{tabular}{llll}
\hline
\textbf{Parameter}       & \textbf{Min}  & \textbf{Step}   & \textbf{Max}  \\
\hline
$n_r$        & 60 & 5 & 400  \\
$d_r$ ($\mu$s)          & 40960 & 5120 & 204800    \\
$s_r$           & 1 & 5 & 50    \\
\hline
\end{tabular}
\end{table}


\subsubsection{Output parameters design}
\textcolor{red}{
As non-linear regions are more difficult to model compared to linear regions, more sample points from the non-linear regions are required in the modelling process. 
Therefore, a good selection of output parameter, which leads to more linear regions, can speed up the modeling process. For our models, there are several options on the output parameters, which can directly or indirectly represent the performance metrics (i.e., throughput and energy) that we focus on.  As in our modeling process, \gls{raw} group duration may be short than the beacon interval, we can use the throughput of the \gls{raw} groups or the throughput of the whole beacon interval as output parameter for throughput model. While the latter one results in more linear output than the former one does, since \gls{raw} group duration varies from data points to data points, while beacon interval is always constant. Similarly, energy consumption per packet or total energy consumption can be considered as output parameter for energy model. While the former one is the ratio between the latter one and the number of successfully transmitted packets, it leads to more more non-linear output than the latter one does. Therefore, we choose throughput of the whole beacon interval and  total energy consumption as output parameters for throughput and energy model, respectively. 
%For the surrogate modeling, Kriging and FLOLA-Voronoi are used for model creation and sampling strategy, respectively. Kriging assumes that data points are spatially correlated, such that points that closer will have comparable performance, i.e., the performance difference between two date points is smaller if they are closer to each other than far away points. A novel sampling strategy called FLOLA-Voronoi [27] isused to select the next design points to improve the modelaccuracy. The FLOLA approach is used for exploiting the non-linear regions, while the Voronoi approach exploresthe sparsely sampled regions, the scores from the FLOLAand  Voronoi  are  combined  to  decide  the  next  samplepoint. In our experiments, 10 new data points are pickedin each interation
}

% \section{Model-Based RAW Optimization Algorithm \label{sec:algorithm}}

This section introduces the \glsreset{moroa}\gls{moroa}. It relies on the same principles as our previously proposed \gls{raw} optimization solutions \gls{taroa}~\cite{Sensor2017} and \gls{etaroa}~\cite{Sensys2017}. As such, it is also traffic-aware and able to adapt to changing traffic conditions. Moreover, \gls{moroa} supports heterogeneous stations with variable \gls{mcs} and packet size.
%In contrast to previous work, \gls{moroa} differentiates in using a model to find the optimal \gls{raw} configuration parameters \textcolor{red}{for different objectives}. 
%This allows it to better estimate the actual performance of a specific \gls{raw} configuration.
\textcolor{red}{
Finally, the extended algorithm presented here differentiates from our original version of \gls{moroa}~\cite{Tian2018} in the fact that it finds the optimal \gls{raw} configuration by optimizing both throughput and energy simultaneously, instead of focusing on a single objective.} 

% This section introduces the RAW optimization problem addressed in this article, and subsequently proposes the Traffic-Adaptive RAW Optimization Algorithm (TAROA). TAROA solves the RAW optimization problem in real-time, and is able to instantaneously adapt to changes in station association and traffic demand. Table~\ref{tab:variables} provides an overview of the variables used in the description of the RAW optimization problem and TAROA.


\subsection{Overview}
As in Section~\ref{sec:SUMO}, we assume an \gls{iot} sensor-based monitoring scenario. However, in contrast to the model presented above, the algorithm is able to handle heterogeneous stations, with variable transmission intervals, \gls{mcs}, and packet sizes. This is achieved, on one hand, by combining different trained models for different types of stations, and on the other hand, by transforming the performance metric output based on traffic conditions. Moreover, the data transmission interval of sensor stations can change over time (e.g., when an environmental event triggers a change in the sensor measurement interval). The goal of \gls{raw} optimization is to assign stations to a set of \gls{raw} groups with appropriate \gls{raw} parameter configurations, in order to achieve the required objective (e.g., maximum throughput, fairness, or minimum energy consumption).  

The proposed algorithm uses only information readily available at the \gls{ap}. The concrete steps are illustrated in Figure~\ref{fig:algorithm}.
\textcolor{red}{The algorithm is executed at each \gls{tbtt} and supports dynamic network conditions (i.e., stations joining or leaving the network) and dynamic traffic (i.e., stations that change their transmission interval).}
First, the \gls{ap} categorizes stations into different groups. 
%\textcolor{red}{This step only needs to be done when the network topology changes}.
In \gls{moroa}, this is based on \gls{mcs} and packet size, as these two factors influence the minimum time needed to successfully transmit a packet, and therefore the optimal slot duration. However, other grouping strategies can be used as well. Second, the \gls{ap} determines the traffic of each station, and selects the stations that are expected to have pending packets to transmit in the next beacon interval. As this information is not readily available to the \gls{ap}, it has to be estimated. We apply a traffic estimation method for \gls{iot} sensor traffic proposed in our previous work~\cite{Sensys2017}. 
Finally, we utilize \gls{raw} performance models $\mathcal{F}\left(n, d, s\right)$ that take as input the number of stations $n$, the group duration $d$, and the number of slots $s$, and gives as output some performance metric $t$ (e.g., throughput, energy). This function serves as the basis for an optimization problem, that optimizes the \gls{raw} parameter decision variables $n$, $d$ and $s$, in order to maximize the selected performance metrics. \textcolor{red}{The optimization is subjected to a set of constraints that in turn ensure the traffic estimation for the following beacon intervals is accurate.} The output of the algorithm is a \gls{raw} configuration consisting of a set of groups, containing for each group a set of assigned stations, group duration, and the number of slots.Note that in the remainder of the description, we assume the use of the surrogate model described in Section~\ref{sec:SUMO}. However, any function $\mathcal{F}\left(n, d, s\right)$ that satisfies the above requirements and that can be calculated in real-time could be used in combination with \gls{moroa}. 



\begin{figure}[t]
  \centering
  \includegraphics[width=0.9\columnwidth]{image/sumo-alg.pdf}
  \caption{ \textcolor{red}{Overview of the steps involved in \gls{moroa}.\label{fig:algorithm}}}
\end{figure}


\subsection{ \textcolor{red}{Multi-objective \gls{raw} parameter optimization with heterogeneous stations}}

We assume a set of stations $\mathcal{N}$ associated with the \gls{ap}. At the start of each beacon interval $b$, stations are split into $k$ distinct clusters, with each cluster $i \in \left[1, k\right]$ consisting of the stations $\mathcal{N}_i \subseteq \mathcal{N}$. This can be achieved using any clustering algorithm, based on a variety of distance metrics. We use standard K-means clustering combined with the packet transmission time as a distance metric. The packet transmission time can be trivially calculated based on \gls{mcs} and packet size, both of which can be monitored at the \gls{ap}. This results in stations with the same \gls{mcs} and average packet size to be clustered together, which is an assumption of our current surrogate model. %\todo{Same packet size is used in our training}. 
In future work, we plan to train models without this assumption, allowing a wider variety of clustering approaches.

Subsequently, by taking the fairness into account, the \gls{ap} determines which stations $\mathcal{N}_i^b \subseteq \mathcal{N}_i$ of each cluster $i$ are predicted to have packets queued for transmission during the next beacon interval $b$. This can be done using our previously proposed traffic estimation method for IEEE~802.11ah~\cite{Sensys2017}. We also define $n_i^b = \left|\mathcal{N}_i^b\right|$ as the number of stations in cluster $i$ predicted to have packets queued for transmission during the next beacon interval $b$. Finally, the algorithm assigns a \gls{raw} group to each cluster $i$. It calculates the number of stations $n_i$ that will be allowed to access the channel, the duration $d_i$ of the group, and in how many slots $s_i$ to split the group. 


\textcolor{red}{Finding the \gls{raw} parameter values that optimize the chosen performance metrics (i.e., maximizing throughput and minimizing energy) can be formulated as a multi-objective optimization problem. %Not only the objectives of a group may not coincide, but also the objectives among different groups can be conflicting due the the airtime constraint.
As the sum of all group durations should not be higher than the beacon interval duration $d_b$, the chosen duration $d_i$ of a cluster $i$ influences the \gls{raw} duration of other clusters. A common approach to solving multi-objective optimization problems is to first generate a Pareto front set from which the most desirable trade-off point is selected. A Pareto front set is a set of solutions where any improvement in one objective results in the worsening of at least one other objective. %In reality, the decision maker is not interested in discovering the whole Pareto front rather than finding only the portion(s) of the front that matches at most his/her preferences. 
There have been many methods suggested for selecting the optimal Pareto front point, such as the weighted global criterion method, weighted sum method, lexicographic method and weighted min-max method \cite{marler2004survey}. These methods allow the user to specify preferences, which may be articulated in terms of goals or the relative importance of different objectives. %Most of these methods incorporate parameters, which are coefficients, exponents, constraint limits, etc. that can either be set to reflect decision-maker preferences, or be continuously altered in an effort to represent the complete Pareto optimal set.
We define an objective function that maximizes throughput and minimizes energy consumption for each group, using a hybrid method combining weighted sum and min-max. However, \gls{moroa} could be trivially extended to use a different optimization objective functions, such as for example airtime fairness.%, or latency minimization can easily be defined as well, as the model can be trivially trained for a variety of metrics. 
The problem can now be formulated as follows:
} 


%This problem has to be solved jointly for all groups, as the chosen duration $d_i$ of a cluster $i$ influences the maximum duration of all other groups (i.e., the sum of all group durations should not be higher than the beacon interval duration $d_b$). 
%As stated in Section~\ref{sec:SUMO}, a variety of objective functions can be defined, as the model can be trivially trained for a variety of metrics such as throughput, latency, energy consumption, and packet loss. As an illustration, we define an objective function that maximizes throughput as well as \textcolor{red}{energy consumption per packet.}
%fairness across groups in terms of packets delivery ratio.  The packet delivery ratio is defined as the ratio  between the total number of packets that are successfully received and the total number of packets generated.
%However, other objectives such as airtime fairness, latency minimization can easily be defined as well. The problem can be formulated as follows:
\textcolor{red}{
\begin{equation} \label{eq:objective}
\max \left( \alpha \times \mathcal{Q}_t - \left(1 - \alpha \right) \times \mathcal{Q}_e \right)
\end{equation}
With:
\begin{equation} \label{eq:objectiveT}
%\mathcal{Q}_t =  \sum\limits_{i=1}^{k}  \frac{\mathcal{F}_t\left(n_i, d_i, s_i\right)}{n_i^b \times l_i}
\mathcal{Q}_t =  \min_{i \in \left[1, k\right]} \left(  \frac{\mathcal{F}_t\left(n_i, d_i, s_i\right)}{n_i^b \times l_i}  \right)
\end{equation}
And
\begin{equation} \label{eq:objectiveF}
%\mathcal{Q}_e =  \max_{i \in \left[1, k\right]} \left(  \frac{ \mathcal{F}_t (n_i, d_i, s_i)  \times l_i}{\mathcal{F}_t\left(n_i, d_i, s_i\right) \times T  \times E_i}  \right)
\mathcal{Q}_e =  \max_{i \in \left[1, k\right]} \left(  \frac{ \mathcal{F}_t (n_i, d_i, s_i)  \times l_i}{\mathcal{F}_t\left(n_i, d_i, s_i\right) \times T  }  \right)
%\mathcal{Q}_e =  \sum \limits_{i=1}^{k}  \frac{\mathcal{F}_e\left(n_i, d_i, s_i\right)}{\mathcal{F}_t\left(n_i, d_i, s_i\right) \times E_i}
\end{equation}
}
Subject to:
\begin{equation} \label{eq:si}
\forall i \in \left[1,k\right]: n_i \leq n_i^{b}
\end{equation}
\begin{equation} \label{eq:sslot}
 \sum\limits_{i=1}^{k} d_i \leq d_b
\end{equation}
\begin{equation} \label{eq:successpro}
 %\mathcal{F}_t \left(n_i, d_i, s_i\right)} {n_i \times l_i}
\forall i \in \left[1,k\right]:  p_i^s < \frac{ n_i \times l_i}{\mathcal{F}_t \left( n_i, d_i, s_i\right)}
\end{equation}
\textcolor{red}{Where $\mathcal{F}_t\left(\cdot\right)$  and $\mathcal{F}_e\left(\cdot\right)$ represent the \gls{raw} model functions that predict throughput and energy consumption}. The variable $l_i$ is the average packet size of stations in cluster $i$. The continuous variable $\alpha \in \left[0, 1\right]$ is a weight used to define the relative importance of both sub-objectives. The parameter $d_b$ represents the duration of the beacon interval $b$. In Eq.~\ref{eq:successpro}, $p_i^s$ represents the successful packet transmission probability of \gls{raw} group $i$. This constraint is required, as the traffic estimation method we use, does not work properly under high packet loss due to contention~\cite{Sensor2017}. As such, we use $p_i^s=0.99$. When using other traffic estimation methods, this constraint may not be needed. 



 \textcolor{red}{
 ${Q}_t$ represents the objective of throughput, while ${Q}_e$ represents the objective of energy consumption per packet. Both objectives try to find a balance among different groups. 
 %As the throughput and energy consumption depend on the \gls{mcs} and packet size, both objectives are normalized to align their valid value range and simplify selection of $\alpha$. 
 $T$ denotes the training time. 
 %$E_i$ represents, when no channel contention occurs, energy consumed by a single packet transmission using the same \gls{mcs} and packet size as group $i$ does. 
 \gls{raw} groups are allowed to use all the airtime (i.e., the entire beacon interval), as represented by Eq. \ref{eq:sslot}. A \gls{raw}-free period can be trivially introduced by changing this constraint.}
%As such, in our algorithm, the airtime used by \gls{raw} groups do not contribute to the objectives.} 
Note that this formulation assumes that stations will only attempt to transmit one packet per beacon interval. This assumption generally holds for sensor scenarios, where the throughput of individual stations is low~\cite{Khorov2015b}. In reality, some stations may have multiple packets queued, especially when the traffic estimator is still learning~\cite{Sensys2017}. However, this has a negligible effect on performance of the algorithm over longer periods. Currently, \gls{moroa} does not consider the required sequentially of \glspl{aid} in the \gls{rps} element, as we consider \gls{aid} reassignment a separate issue left for future work.


%The normalized throughput represent the ratio between packet received and transmitted, and normalized energy consumption per packet is calculated as the ratio between energy consumption per packet and $E_i$.  $E_i$ represents, when no channel contention occurs, energy consumed by a single packet transmission for a station having same the \gls{mcs} and packet size as group $i$ does.


%\textcolor{red}{For the moment, we slightly modify the RPS element to assign stations to the same RAW group without the sequential AID constraint, AID dynamic re-assignment is considered future work to fully stick to the 802.11ah requirement.}
%To be able to assign modifying the PS-Poll frame structure in the 802.11ah standard Since only stations with sequential AID can be assigned to the same group, AID re-assignment to further optimize latency as well as throughput is considered future work.}
The formulated problem is a non-linear constrained optimization problem with integer decision variables (i.e., $n_i$, $d_i$, and $s_i$). This can for example be solved using genetic algorithms. A relaxed version of the problem with continuous decision variables could alternatively be solved using for example the Interior-Point method, in combination with a rounding strategy to convert the resulting continuous decision variable values to integers. Moreover, by taking the characteristic 802.11ah into account, the value of $n_i$ and $k$ can be further limited, reducing the number of potential solutions and therefore the solving time. The sum of $n_i$ for all the RAW groups is limited to the maximum number of packet $n_{max}^b$ that can be successfully transmitted during one beacon interval. Due to the relatively low data rates supported by 802.11ah (i.e., up to 7.8~Mbps for a 2~MHz bandwidth), $n_{max}^b$ ranges from around $10$ to $100$ in practice. \textcolor{red}{ For homogeneous networks, we can use the corresponding $n_{max}^b$ based on the used \gls{mcs}. For heterogeneous networks in which multiple \gls{mcs}s are used, we can use the $n_{max}^b$ (equal to 100) of \gls{mcs} 8 for a 2~MHz bandwidth. In the \gls{rps} element, each \gls{raw} group definition requires between 3 and 12 bytes. As the maximum size of the \gls{rps} elements is $256$~bytes, the number of groups $k$ cannot be higher than 85.
%The number of groups $k$ cannot be higher than 42, as the maximum size of the \gls{rps} elements is $256$~bytes and at least $6$~bytes are needed per group. 
Based on the specification of \gls{raw}, either the parameter $s_i$ is at most 64 and $d_i$ is at most $1990.4$~ms, or the parameter $s_i$ is at most 8 and $d_i$ is at most $1969.12$~ms
}

% \subsection{\gls{raw} parameter optimization with heterogeneous stations}

% We assume a set of stations $\mathcal{N}$ associated with the \gls{ap}. At the start of each beacon interval $b$, stations are split into $k$ distinct clusters, with each cluster $i \in \left[1, k\right]$ consisting of the stations $\mathcal{N}_i \subseteq \mathcal{N}$. This can be achieved using any clustering algorithm, based on a variety of distance metrics. We use standard K-means clustering combined with the packet transmission time as a distance metric. The packet transmission time can be trivially calculated based on \gls{mcs} and packet size, both of which can be monitored at the \gls{ap}. This results in stations with the same \gls{mcs} and average packet size to be clustered together, which is an assumption of our current surrogate model. %\todo{Same packet size is used in our training}. 
% In future work, we plan to train models without this assumption, allowing a wider variety of clustering approaches.

% Subsequently, by taking the fairness into account, the \gls{ap} determines which stations $\mathcal{N}_i^b \subseteq \mathcal{N}_i$ of each cluster $i$ are predicted to have packets queued for transmission during the next beacon interval $b$. This can be done using our previously proposed traffic estimation method for IEEE~802.11ah~\cite{Sensys2017}. We also define $n_i^b = \left|\mathcal{N}_i^b\right|$ as the number of stations in cluster $i$ predicted to have packets queued for transmission during the next beacon interval $b$. Finally, the algorithm assigns a \gls{raw} group to each cluster $i$. It calculates the number of stations $n_i$ that will be allowed to access the channel, the duration $d_i$ of the group, and in how many slots $s_i$ to split the group. Finding the \gls{raw} parameter values that maximize the chosen performance metric can be formulated as an optimization problem. This problem has to be solved jointly for all groups, as the chosen duration $d_i$ of a cluster $i$ influences the maximum duration of all other groups (i.e., the sum of all group durations should not be higher than the beacon interval duration $d_b$).

% As stated in Section~\ref{sec:SUMO}, a variety of objective functions can be defined, as the model can be trivially trained for a variety of metrics such as throughput, latency, energy consumption, and packet loss. As an illustration, we define an objective function that maximizes throughput as well as fairness across groups in terms of packets delivery ratio. The packet delivery ratio is defined as the ratio  between the total number of packets that are successfully received and the total number of packets generated.
% However, other objectives such as airtime fairness, latency minimization, or energy efficiency can easily be defined as well. The problem can be formulated as follows:
% \begin{equation} \label{eq:objective}
% \max \left( \alpha \times \mathcal{Q}_p + \left(1 - \alpha \right) \times \mathcal{Q}_f \right)
% \end{equation}
% With:
% \begin{equation} \label{eq:objectiveT}
% \mathcal{Q}_p =  \sum\limits_{i=1}^{k}  \frac{\mathcal{F}_t\left(n_i, d_i, s_i\right)}{n_i^b \times l_i}
% \end{equation}
% And
% \begin{equation} \label{eq:objectiveF}
% \mathcal{Q}_f =  \min_{i \in \left[1, k\right]} \left(  \frac{ \mathcal{F}_t (n_i, d_i, s_i) }{n_i^b \times l_i}  \right)
% \end{equation}
% Subject to:
% \begin{equation} \label{eq:si}
% \forall i \in \left[1,k\right]: n_i \leq n_i^{b}
% \end{equation}
% \begin{equation} \label{eq:sslot}
%  \sum\limits_{i=1}^{k} d_i \leq d_b
% \end{equation}
% \begin{equation} \label{eq:successpro}
%  %\mathcal{F}_t \left(n_i, d_i, s_i\right)} {n_i \times l_i}
% \forall i \in \left[1,k\right]:  p_i^s < \frac{ n_i \times l_i}{\mathcal{F}_t \left( n_i, d_i, s_i\right)}
% \end{equation}
% Where $\mathcal{F}_t\left(\cdot\right)$ represents the \gls{raw} model function that calculates throughput. The variable $l_i$ is the average packet size of stations in cluster $i$. The continuous variable $\alpha \in \left[0, 1\right]$ is a weight used to define the relative importance of both sub-objectives. The parameter $d_b$ represents the duration of the beacon interval $b$. In Eq.~\ref{eq:successpro}, $p_i^s$ represents the successful packet transmission probability of \gls{raw} group $i$. This constraint is required, as the traffic estimation method we use, does not work properly under high packet loss due to contention~\cite{Sensor2017}. As such, we use $p_i^s=0.99$. When using other traffic estimation methods, this constraint may not be needed.

% ${Q}_f$ represents the fairness objective, while ${Q}_p$ represents throughput. Both objectives are normalized as to align their valid value range and simplify selection of $\alpha$. Note that this formulation assumes that stations will only attempt to transmit one packet per beacon interval. This assumption generally holds for sensor scenarios, where the throughput of individual stations is low~\cite{Khorov2015b}. In reality, some stations may have multiple packets queued, especially when the traffic estimator is still learning~\cite{Sensys2017}. However, this has a negligible effect on performance of the algorithm over longer periods. Currently, \gls{moroa} does not consider the required sequentiality of \glspl{aid} in the \gls{rps} element, as we consider \gls{aid} reassignment a separate issue left for future work.

% %\textcolor{red}{For the moment, we slightly modify the RPS element to assign stations to the same RAW group without the sequential AID constraint, AID dynamic re-assignment is considered future work to fully stick to the 802.11ah requirement.}
% %To be able to assign modifying the PS-Poll frame structure in the 802.11ah standard Since only stations with sequential AID can be assigned to the same group, AID re-assignment to further optimize latency as well as throughput is considered future work.}
% The formulated problem is a non-linear constrained optimization problem with integer decision variables (i.e., $n_i$, $d_i$, and $s_i$). This can for example be solved using genetic algorithms. A relaxed version of the problem with continuous decision variables could alternatively be solved using for example the Interior-Point method, in combination with a rounding strategy to convert the resulting continuous decision variable values to integers. Moreover, by taking the characteristic 802.11ah into account, the value of $n_i$ and $k$ can be further limited, reducing the number of potential solutions and therefore the solving time. The sum of $n_i$ for all the RAW groups is limited to the maximum number of packet $n_{max}^b$ that can be successfully transmitted during one beacon interval. Due to the relatively low data rates supported by 802.11ah (i.e., up to 7.8~Mbps for a 2~MHz bandwidth), $n_{max}^b$ ranges from around $10$ to $100$ in practice. The number of groups $k$ cannot be higher than 42, as the maximum size of the \gls{rps} elements is $256$~bytes and at least $6$~bytes are needed per group. Based on the specification of \gls{raw}, the parameter $s_i$ is at most 64 and $d_i$ is at most $1990.4$~ms.


%\textcolor{red}{TODO: we should give some indication that solving the problem is fast, as the values that $n_i$ and $s_i$ can take that would lead to good results are limited. Can we show this using some calculation?}

%Where $\mathcal{F}_i(\cdot)$ represents the built RAW model for type $i$ stations. $L_i$ denotes the packet length of stations assigned to group $i$, $\mathcal{P}_i(\cdot)$ calculates the estimated traffic load of RAW group $i$ based on the consumption that the station is estimated to has one packet when it is assigned to a RAW group. This consumption generally holds given the sensor use case considered in this paper, a station will at most transmit one packet per beacon interval (generally set to 200 ms). Although during the learning phase of traffic estimation, more than one packet may be accumulated at the transmission queue of the station, this can be well solved via again assigning the station to a RAW group in the next beacon interval by utilizing the ''more data" filed, more details can be found in the E-TAROA algorithm \cite{Sensys2017}. Eq. (\ref{eq:objectiveT}) aims for the normalized throughput of the network, while Eq. (\ref{eq:objectiveF}) targets at the fairness among different types of stations in terms of the percentage of traffic that can be delivered to the AP. Straightforwardly, the stations assigned to RAW group should not be larger than the maximal number of station having packets to send, as indicated in Eq. (\ref{eq:si}), $s_i < s_i^{max}$ occurs  the network is under the overloaded traffic condition. Since there is no overlap between different different RAW groups, as Eq \ref{eq:sslot} suggests, the total RAW duration should not exceed beacon interval $B$. The objective function Eq. (\ref{eq:objective}) is adjustable by tuning the value of $\alpha$, a larger $\alpha$ indicates the throughput is more important in the evaluated scenarios, a smaller $\alpha$ can be used when fairness has the high priority.

%The RAW optimization is an integer linear programming (ILP) problem, which can be solved online or offline. The online method is suitable for the small scale network(i.e., a few station types and station number stations), since there is a very limited time available before the  AP sends the beacon frame carrying RAW configuration information. For offline, the optimal RAW configuration can be calculated in advance for all the possible combination $\{ n_1^{B}, n_2^{B}, \cdot, n_n^{B} \}$. With the pre-calculated RAW configuration, the AP is able to estimate the traffic of stations and group the stations in an optimal way in a very short time. 








% \section{Performance Evaluation and Discussion \label{sec:evaluation}}

This section presents the evaluation results of the \gls{raw} performance of surrogate model and the \gls{moroa} \gls{raw} optimization algorithm. First, the simulation setup is discussed. Second, the accuracy of the surrogate model is compared to simulation results. Finally, \gls{moroa} is evaluated and compared to state of the art \gls{raw} algorithms, as well as the traditional \gls{edca} method of IEEE~802.11.

\begin{figure}[t]
  \centering
  \includegraphics[width=0.75\columnwidth]{image/throughput_cv}
  \caption{\textcolor{black}{Cross validation score of the surrogate models as a function of the number of training \textcolor{red}{sample data points}.} \label{fig:sumo-iteraton}}
\end{figure}

\subsection{Simulation setup}


% All evaluations are performed using our previously developed 802.11ah ns-3 module~\cite{WNS32016}, based on ns-3 version 3.23. We consider the same IoT scenario as described in section \ref{sec:algorithm},  the default PHY and MAC layer parameters used in our simulation are shown in Table~\ref{tab:parameters}. Concretely,  we consider two types of stations (i.e., HT and LT station as defined in section \ref{sec:SUMO}) co-exists in the one network, each station randomly choose a value X (second) from the range [1, 10] as its transmission interval. We use $\eta$to tune the percentage of the number of each type of station in the whole network, $\eta$, and ($100-\eta$) stations belongs to HT and LT respectively, $\eta=\left[0, 0.25, 0.50, 0.75, 1\right]$~\% are evaluated in the experiment. 
   

% RAW performance is evaluated in terms of two metrics: throughput, fairness, and latency. Throughput is calculated as the average number of successfully received payload bytes by the AP per second. Fairness among HT and LT stations are evaluated using Jain's fairness index \cite{jain1984quantitative}. Latency is defined as the average time between a packet entering the transmit queue of the station and being received by the AP. Each simulation runs 600~s . 


All evaluations are performed using our previously developed IEEE~802.11ah ns-3 module~\cite{WNS32016}, based on ns-3 version~3.23. We consider the same \gls{iot} scenario as described in Section~\ref{sec:algorithm}. The same default PHY and MAC layer parameters used as shown in Table~\ref{tab:ns3 parameters}. We consider both homogeneous and heterogeneous scenarios. Homogeneous scenarios are used to validate the surrogate model and compare to \gls{etaroa}~\cite{Sensys2017}. In heterogeneous scenarios, half of the stations use the high-throughput (HT) settings and half of them use the low-throughput settings (LT) listed in Table~\ref{tab:ns3 parameters}. The data transmission interval of each station is selected uniformly at random from the interval $\left[1, 10\right]$~seconds.

%RAW performance is evaluated in terms of three metrics: throughput, fairness, and latency. Throughput is calculated as the average number of successfully received payload bytes by the \gls{ap} per second. Fairness between HT and LT stations in terms of packet delivery ratio is evaluated using Jain's fairness index. Latency is defined as the average time between a packet entering the transmit queue of the station and being received by the \gls{ap}. 

\textcolor{black}{\gls{raw} performance is evaluated in terms of two metrics: throughput and energy. Throughput is calculated as the average number of successfully received application payload bits by the \gls{ap} per second. Energy is considered as the total energy consumption averaged over the number of successfully received packets.} Each simulation runs for 600~seconds, and all results are averaged over 10 iterations, with the variability of results over these iterations quantified using the standard deviation (SD).



% Run original E-TAROA (one station per slot) until the estimation is stable, then switch to E-TAROA(SUMO). \textcolor{black}{when should the switch happen?}
% \\
% How to switch:\\
% 1. reassign AID based on the transmission interval. Cluster stations based on their AIDs, resulting in:\\
% Interval 1, $n_1$ stations \\
% Interval 2, $n_2$ stations \\
% ... ...\\
% Interval k, $n_k$ stations \\

\subsection{SUMO model validation}




%        \includegraphics[clip, trim=0.5cm 11cm 0.5cm 11cm, width=1.00\textwidth]{gfx/BI-yourfile.pdf}

\begin{figure}[t]
  \centering
    \includegraphics[width=0.75\columnwidth]{image/model_error}
  \caption{\textcolor{black}{Performance comparison between the surrogate model and simulation results for 5000 random test data points. %, sorted in ascending order in terms of estimation error
  }\label{fig:sumo-test-data}}
\end{figure}

In this section we evaluate the training convergence of the surrogate model, as well as its accuracy compared to simulation results. Figure~\ref{fig:sumo-iteraton} plots the model's cross validation score as a function of the number of training samples used. The cross validation score provides a measure for the accuracy of the resulting model \textcolor{red}{on predication of \gls{raw} performance using the IEEE~802.11ah ns-3 simulator}. A consistently low score signifies that the training process has converged. 
%Based on this graph, we can conclude that for both the model with high-throughput (HT) and low-throughput (LT) stations, convergence occurs after around $1700$ training samples have been used.
The results show, for the throughput models of both HT and LT stations, that the convergence occurs after around $1700$ training samples have been used. This comes down to about $0.0074$\% of all data points in the input space (i.e., $2.3 \times 10^7$), and about $6.8$\% of the reduced data space (i.e., $25047$) from which samples were drawn during training. The convergence of energy models happens after 1900 and 2500 training samples have been used for energy models of  HT and LT stations, respectively. The training of the HT throughput model and LT energy model stopped after 2370 and 2400 \st{iterations}  \textcolor{red}{training samples} respectively, as it satisfied the cross-validation stop conditions, i.e., 10 consecutive cross-validation scores (2 digits of precision) remain the same. The training of the other models stopped after the maximum number of 2500 \st{iterations} \textcolor{red}{training samples}  , having achieved 15 consecutive cross-validation scores between $0.11$ and $0.12$ for the LT throughput model, and 9 consecutive cross-validation scores between $0.054$ and $0.058$ for the HT energy model.  The cross-validation scores decrease as more training samples are used, is due to the sampling strategy FLOLA-Voronoi. It selects points from non-linear regions and the sparsely sampled regions, making the model more and more accurate. It should be noted that the cross-validation scores fluctuate sometimes. This is because new training sample points are selected from an untouched region, and the surrogate model is  recalculated to consider this change.

%In order to ensure no over-fitting occurred, the surrogate model also provides accurate results for data points outside the reduced input space (cf. Table~\ref{tab:sumo parameters}), Figure~\ref{fig:sumo-test-data} plots the absolute error (in terms of throughput) of the surrogate model compared to simulation results. In total, 2070 random data points were generated from all $2.3 \times 10^7$ possible points. The figure also plots the actual simulated throughput of each point, to characterize the significance of the error. For the HT model, the absolute throughput estimation error stays below 0.02~Mbps for the first 2000 points (i.e., $96.6$\% of all data points). The absolute error only goes above $0.1$~Mbps for 3 data points (i.e., $0.14$\%). Similarly, for the LT model, the throughput estimation error stays below $0.004$~Mbps for the first 2000 data points, and only grows above $0.01$~Mbps for the worst 11 data points. In terms of the relative error (i.e., ratio between absolute throughput error and simulation results, not depicted) of the surrogate model compared to simulations, 95\% of all data points have an error below 6.6\% and 5.4\% for the HT and LT model respectively. For both models, 98\% of the data points have a relative error below 10\%. These results validate the ability of the surrogate model to estimate \gls{raw} performance accurately, even for points outside of the reduced input data space used for initial training.

\textcolor{black}{In order to ensure no over-fitting occurred, we evaluated if the surrogate model also provides accurate results for data points outside the reduced input space (cf. Table~\ref{tab:sumo parameters}). Figure~\ref{fig:sumo-test-data} plots the \gls{cdf} of the relative error (i.e., ratio between absolute error and simulation results). In total, 5000 random data points were generated from all $2.3 \times 10^7$ possible points. It shows around 95\% and 98\% of the test data points have a relative error less than 10\% for the throughput and energy models for LT stations, respectively. Similar results are obtained for the HT stations. 90\% of the test data points have a relative error less than 7\% for both the throughput and energy models of LT stations. For the throughput and energy models of HT stations, this is respectively 7\% and 5\%. These results validate the ability of the surrogate model to estimate \gls{raw} performance accurately, even for points outside of the reduced input data space used for initial training.}


\subsection{\gls{raw} configuration analysis}

% \begin{figure}[t]
%     \centering
%     \subfloat[LT, energy per packet \label{fig:traffic-pattern-throughput}]{\includegraphics[width=0.35\textwidth]{image/LT-pareto-front-energyPerPacket}}\\
%     \subfloat[Energy consumption per packet \label{fig:traffic-pattern-throughput}]{\includegraphics[width=0.35\textwidth]{image/HT-pareto-front-energyPerPacket}}
%   \caption{\gls{raw} configurations for top 10\% highest energy efficiency. \label{fig:ana-energy}}
% \end{figure}

\textcolor{black}{In this section, based on the built model, we analyze the impact of different \gls{raw} configurations on throughout and energy consumption, and derive the optimal \gls{raw} configuration. For each number of stations, we first find the highest output $t_{max}$ from the built model $\mathcal{F}_t\left(\cdot\right)$, and find $e_{min}$ from the model $\mathcal{F}_e^p \left(\cdot\right)$, which is calculated as follows:
\begin{equation} \label{eq:objectiveF}
\mathcal{F}_e^p (n_r, d_r, s_r) =     \frac{ \mathcal{F}_e (n_r, d_r, s_r) \times l_r }{\mathcal{F}_t\left(n_r, d_r, s_r\right) \times \textcolor{red}{T_c} }  
\end{equation}
The model $\mathcal{F}_e^p (n_i, d_i, s_i)$ predicts the energy consumption per packet for a \gls{raw} group $r$ with duration $d_r$, consisting of $s_r$ slots, and with $n_r$ stations assigned to it. Subsequently, we select the \gls{raw} configurations that deviate at most 10\% from the optimal throughput $t_{max}$ and the optimal energy consumption $e_{min}$.
%that lead to near optimal performance, i.e., throughput higher than 90\% of $t_{max}$, or energy consumption lower than 110\% of $e_{min}$.
}

\begin{figure*}[t]
    \centering
    \subfloat[Throughput \label{fig:traffic-pattern-throughput}]{\includegraphics[width=0.45\columnwidth]{image/LT-pareto-front-throughput}}
    \subfloat[Energy consumption per packet \label{fig:traffic-pattern-energy}]{\includegraphics[width=0.45\columnwidth]{image/LT-pareto-front-energyPerPacket}}
  \caption{\gls{raw} configurations that deviate at most 10\% from the optimal throughput or energy consumption. \label{fig:ana-throughput}}
\end{figure*}

\begin{figure*}[!h]
    \centering
    \subfloat[100 stations \label{fig:pareto-LT}]{\includegraphics[width=0.45\columnwidth]{image/pareto-front101}}
    \subfloat[400 stations \label{fig:pareto-HT}]{\includegraphics[width=0.45\columnwidth]{image/pareto-front401}}
    \caption{Pareto front of throughput and energy per station, pareto front points are highlighted in red. \label{fig:sumo-pareto}}
\end{figure*}

\textcolor{black}{Figure~\ref{fig:ana-throughput}  depicts such \gls{raw} configurations for 100, 200 and 300 LT stations, respectively. As the figures show, if the number of stations is relatively small (i.e., 100), many different \gls{raw} configurations lead to near optimal throughput. This is because channel contention is less fierce in such a scenario, and the \gls{raw} configuration thus has less effect on throughput. Even without \gls{raw} (i.e., one \gls{raw} slot), quite high performance can already be obtained, while a larger number of \gls{raw} slots is definitely needed when the number of contending stations increases to 200 or 300). It should be noted that the near optimal number of \gls{raw} slots for throughput is discontinuously distributed. For example, for \gls{raw} duration 204.8 ms and 200 stations, a number of slots in the intervals 9--16, 21--26 and 40--50 achieve the highest throughput. This is because cross-slot boundary transmissions were disabled, which leads to varying amounts of wasted airtime (i.e., holding time) \cite{Evgeny2018}. Different from throughput, there are fewer near optimal \gls{raw} configurations for energy.}



\textcolor{black}{There exist some \gls{raw} configurations that can maximize throughput and minimize energy simultaneously, as depicted in Figure~\ref{fig:sumo-pareto}. This figure illustrates the Pareto front of throughput and energy consumption per packet for 100 and 400 LT stations. As shown, the 400 stations have only one Pareto front solution, i.e., the objective of maximizing throughput and minimizing energy consumption can be achieved simultaneously. While for 100 stations there are several Pareto front solutions. Among them, a solution can be chosen based on the relative importance of each of the two objectives, using a multi-objective \gls{raw} optimization algorithm such as \gls{moroa}.}

%For a certain slot number $s_r$ , once a high throughput can be achieved, a similar throughput can also be obtained with slot number $s_r^{'}$ larger than $s_r$. The reason is, $s_r^{'}$ number leads to less number of stations assigned to one \gls{raw} slot, therefore mitigate the channel contention, and maintain the throughput, or increase the throughput if there are more packets need to be sent. However, there are several exceptions. First of all, if the number of slots is too big, it may cause the \gls{raw} slot duration too short to transmit a single packet. Second,  \todo{the effect of non-cross slot boundary}. While for cross slot boundary, the performance fluctuation does not occur  \cite{WoWMoM2016} and \cite{Zheng2014}. Second, it can be observed that the normalized throughput increases in a fluctuating way as the duration of the RAW slot increases. The reason is that a varying number of mini-slots can be wasted due to the RAW slot handover between groups.


% \begin{enumerate}
% \item There is a set of \gls{raw} slot numbers that can achieve high performance. 
% \item For shorter \gls{raw} duration, there is less options on the \gls{raw} slot number.
% \todo{affected by non cross slot boundary}
% \item For a certain number of stations, there is a requirement on the minimal \gls{raw} duration, denoted as \gls{d_r^{min}}. Any \gls{raw} duration larger than \gls{d_r^{min}}, can leads to higher throughput, with more options on the \gls{raw} slot number.
% \todo{affected by non cross slot boundary}
% \end{enumerate}

% \textbf{findings for energy from simulation results}
% \begin{enumerate}
% \item Energy efficiency increase as \gls{raw} duration increase.
% \item Energy efficiency increase as \gls{raw} duration increase.
% \end{enumerate}

% \textbf{findings for total energy from \ref{fig:sumo-energy}}
% \begin{enumerate}
% \item Large \gls{raw} duration has high energy efficiency .
% \item Requirement on the minimal \gls{raw} duration.
% \item Requirement on the minimal \gls{raw} \gls{raw} slot number for relative small \gls{raw} duration. Since too few slot number results in high contention.
% \item No Requirement on the minimal \gls{raw} \gls{raw} slot number for large \gls{raw} duration. because that the \gls{raw} slot duration is big enough.
% \item Number of good \gls{raw} performance configuration decrease as station number increase.
% \end{enumerate}


% \textbf{findings for energy on per packet, similar to throughput}


% \textbf{findings for throughput, HT}
% \begin{enumerate}
%     \item 
% \end{enumerate}


% \textbf{summary of the figures, when there is not airtime limit}
% \begin{enumerate}
%     \item when there is not airtime limit \\
%     The packet deliver ratio and energy efficiency overlap, there exists some \gls{raw} configurations that can maximize both objectives simultaneously.  Except (findings from 2070 simulations)
%     \begin{enumerate}
%         \item exception, HT, 204800 us and slot = 1, always got good throughput
%         \item exception, HT, 204800 us and slot = 1, always got good energy from 61 to 336 stations, then slot = 50 get good energy efficiency. 336 stations is a break point, where the achievable throughput start to decrease.
%          \item exception, LT, 204800 us and slot = 1, always got energy until 216 stations. Then 50 stations get good energy efficiency.
%          \item exception, LT, 204800 us and slot = 1, always got throughput until 216 stations. Then 50 stations get good throughput.

         

%     \end{enumerate}
%     \item when airtime is an valuable resource,  i.e., throughput is considered with packet deliver ratio higher than 99\%  \\
%     there is a tradeoff.
% \end{enumerate}













\subsection{Homogeneous stations}

\begin{figure*}[t]
    \centering
    \subfloat[LT, Throughput \label{fig:traffic-pattern-throughput}]{\includegraphics[width=0.45\columnwidth]
    {image/algorithm/throughput-aidaddress-alpha-1-0}} 
    \subfloat[LT, Energy consumption \label{fig:traffic-pattern-throughput}]{\includegraphics[width=0.45\columnwidth]
    {image/algorithm/Energy_aidaddre_ncr-alpha-1-0}} \\
    \subfloat[HT, Throughput \label{fig:traffic-pattern-throughput}]{\includegraphics[width=0.45\columnwidth]
    {image/algorithm/throughput-homo-nohidden-etaroa-aidaddress-alpha-1-0-HT}} 
    \subfloat[HT, Energy consumption \label{fig:traffic-pattern-throughput}]{\includegraphics[width=0.45\columnwidth]
    {image/algorithm/EnergyHomo_etraoa_aidaddre_ncr-alpha-1-0-HT}}
  \caption{\textcolor{red}{Performance comparison between \gls{moroa} and \gls{etaroa} using $\alpha=0, 1$  for the LT and HT scenarios with different traffic loads and station counts} \label{fig:sumo-home}
  }
\end{figure*}

% \begin{figure}[t]
%   \centering
%     \includegraphics[width=0.75\columnwidth]{image/withalpha}
%   \caption{\textcolor{black}{The performance of \gls{moroa} as a function of different $\alpha$
% }\label{fig:homo-alpha}}
% \end{figure}

\begin{figure*}
    \centering
    \subfloat[LT, Throughput \label{fig:homo-alpha-throughput}]{\includegraphics[width=0.45\columnwidth]
    {image/withalpha-throughput}} 
    \subfloat[LT, Energy consumption \label{fig:homo-alpha-energy}]{\includegraphics[width=0.45\columnwidth]
    {image/withalpha-energy}} \\
    \subfloat[HT, Throughput \label{fig:homo-alpha-throughput}]{\includegraphics[width=0.45\columnwidth]
    {image/algorithm/withalpha-throughput-review}} 
    \subfloat[LT, Energy consumption \label{fig:homo-alpha-energy}]{\includegraphics[width=0.45\columnwidth]
    {image/algorithm/withalpha-energy-review}} 
  \caption{\textcolor{red}{Performance of \gls{moroa} as a function of different $\alpha$} \label{fig:homo-alpha}
  }
\end{figure*}







%Compare original E-TAROA (one station per slot) and revised E-TAROA (using SUMO modeling) in terms of throughput, latency and energy consumption. 
%
%In this section we evaluate the performance of \gls{moroa} for homogeneous stations (in terms of \gls{mcs} and packet size) for a variety of traffic loads and station counts. Three different total traffic loads are simulated for the LT scenario, i.e., $T$ = \{0.095, 0.11, 0.15\}~Mbps.  Given the packet payload size and data rate, the maximum throughput that can be achieved is about 0.124~Mbps for LT stations. As such, $T = 0.15$~Mbps represents a near-saturated state, $T = 0.11$~Mbps represents a medium traffic load and $T = 0.095$~Mbps results in low traffic load. %Each station randomly chooses its transmission transmission interval, and the ratio between maximal and minimal transmission interval is up to 20 \cite{Sensor2017}. 
%As \gls{etaroa} was already shown to outperform \gls{edca} for homogeneous stations, we use \gls{etaroa} as a benchmark in this section~\cite{Sensys2017}.


In this section, we evaluate the performance of \gls{moroa} for homogeneous stations (in terms of \gls{mcs} and packet size) with the multi-objective function weight $\alpha$  (cf. Eq.~\ref{eq:objective}) equals $0$ (i.e., pure energy optimization) and $1$ (i.e., pure throughput optimization) for a variety of traffic loads and station counts. Two different total traffic loads are simulated, i.e., $T$ = \{0.11, 0.15\}~Mbps \textcolor{red}{ for the LT scenario, and $T$ = \{0.75, 1.10\}~Mbps for the HT scenario}.  Given the packet payload size and \gls{mcs}, the maximum throughput that can be achieved is about 0.124~Mbps \textcolor{red}{and 0.90~Mbps for LT and HT stations respectively}. As such, $T = 0.15$~Mbps \textcolor{red}{and $T = 1.10$~Mbps} represent a saturated state, $T = 0.11$~Mbps \textcolor{red}{and $T = 0.75$~Mbps} represents a medium to high traffic load. Each station is assigned a random transmission interval, with the ratio between maximum and minimum transmission interval at most 20. 
As \gls{etaroa} was already shown to outperform \gls{edca} and other \gls{raw} optimization algorithms for homogeneous scenarios, we use \gls{etaroa} as a benchmark in this section~\cite{Sensys2017}.

%Figure~\ref{fig:sumo-home} depicts performance (i.e., throughput and latency) of \gls{moroa} and \gls{etaroa}. It clearly shows that, compared to \gls{etaroa}, \gls{moroa} improves the throughput under near saturated traffic conditions for any number of stations. At low traffic loads, \gls{moroa} and \gls{etaroa} have the same throughput for any number of stations. Under medium traffic conditions, \gls{moroa} also improves performance for a low number of stations. More importantly, better latency performance is achieved in denser networks. For a low traffic load (i.e., $0.095$~Mbps), \gls{moroa} and \gls{etaroa} have  nearly the same latency when there are less than 1024 stations. However, for 2048 stations, latency of \gls{moroa} is about $32.5$\% lower. For higher traffic loads, \gls{moroa} has better latency even in less dense networks (i.e., starting from 512 and 128 stations for traffic load $0.11$ and $0.15$~Mbps respectively). The above results reveal that, by using surrogate modeling, stations are assigned to RAW groups in a more optimal way. This is due to the fact that \gls{etaroa} derives its optimal \gls{raw} parameters from a model under saturated state as an approximation, rather than optimizing them for different traffic conditions as the surrogate model does. This results in better overall performance in terms of throughput and latency.

Figure~\ref{fig:sumo-home} depicts performance in terms of throughput and energy consumption of \gls{moroa} and \gls{etaroa}. It clearly shows that, compared to \gls{etaroa}, \gls{moroa} can tune the values of $\alpha$ to optimize the required performance objective. For $\alpha = 1$, \gls{moroa} improves the throughput  up to 5.3\% under saturated traffic conditions for any number of stations, \textcolor{red}{for both LT and HT stations}. Under medium traffic conditions, \gls{moroa} also improves performance. While \gls{moroa} consumes more energy than \gls{etaroa} does for both saturated and medium traffic conditions, up to 30\%. 
%Moreover, \gls{etaroa} has the similar power consumption for both traffic load, while the power consumption of \gls{moroa} increases when the network traffic load changes. 
Conversely, when $\alpha$ is $0$, \gls{moroa} consumes up to 22.6\% less energy at the cost of 4.4\% throughput decrease \textcolor{red}{for LT stations, and up to 32.4\% less energy at the cost of 3.26\% throughput decrease for HT stations}. This is due to the fact that \gls{etaroa} derives its optimal \gls{raw} parameters from a model of throughput under saturated state as an approximation, rather than optimizing them for different traffic conditions and objectives as the surrogate model does. This results in better overall performance in terms of throughput and energy efficiency. Figure \ref{fig:homo-alpha} zooms into \gls{moroa} performance with different $\alpha$ for 1024 \textcolor{red}{LT and 1024 HT stations,  with traffic load 0.15 Mpbs and 1.10 Mbps respectively}. It clearly shows, Compared to \gls{etaroa}, that by adjusting the value of $\alpha$, \gls{moroa} can either obtain higher throughput or lower power consumption. Moreover, the energy consumption is significantly affected by $\alpha$, up to 22.6\% \textcolor{red}{ and 21.2\%} can be saved \textcolor{red}{for LT and HT stations respectively} . While $\alpha$ has a small impact on throughput, only 4.5\% \textcolor{red}{and 4.0\%} throughput can be improved \textcolor{red}{for LT and HT stations respectively}. \textcolor{black}{
The above results reveal that, by using surrogate modeling, stations can be assigned to RAW groups in a more optimal way for the desired performance objective by adapting the $\alpha$ value. Moreover, these results show that by employing multi-objective optimization, a small reduction in throughput can lead to a significant improvement in terms of energy efficiency.}
%This is due to the fact that \gls{etaroa} derives its optimal \gls{raw} parameters from a model under saturated state as an approximation, rather than optimizing them for different traffic conditions as the surrogate model does. This results in better overall performance in terms of throughput and latency.



\subsection{Heterogeneous stations}
%Compare original E-TAROA (one station per slot) and revised E-TAROA (using SUMO modeling) in terms of throughput, latency and energy consumption. 
% 


%\begin{figure*}[t]
%    \subfloat[Throughput \label{fig:throughput-alpha}]{\includegraphics[width=0.5\textwidth]{image/algorithm/throughput-alpha}}
%    \subfloat[Fairness \label{fig:fairness-alpha-percent}]{\includegraphics[width=0.5\textwidth]{image/algorithm/fariness-alpha-percentage}}
%  \caption{Performance of \gls{moroa} with different $\alpha$ with heterogeneous stations. \label{fig:sumo-alpha}}
%\end{figure*}

% \begin{figure}[t]
%     \centering
%     \subfloat[Throughput \label{fig:throughput-alpha-0}]{\includegraphics[width=0.35\textwidth]{image/algorithm/final-throughput-merge-alpha-0}}\\
%     \subfloat[Energy consumption \label{fig:delay-alpha-0}]{\includegraphics[width=0.35\textwidth]{image/algorithm/final-fariness-merge-alpha-0percentage}}
%   \caption{\textcolor{black}{Performance comparison between \gls{moroa} and \gls{edca}, for a heterogeneous network.} \label{fig:sumo-throughput-csma}}
% \end{figure}

\begin{figure*}[t]
    \centering
    \subfloat[Throughput, LT \label{fig:throughput-alpha-0}]{\includegraphics[width=0.45\columnwidth]{image/algorithm/Final-throughput-alpha-LT}}
    \subfloat[Throughput, HT \label{fig:delay-alpha-0}]{\includegraphics[width=0.45\columnwidth]{image/algorithm/Final-throughput-alpha-HT}} \\
     \subfloat[Energy consumption, LT \label{fig:throughput-alpha-0}]{\includegraphics[width=0.45\columnwidth]{image/algorithm/Final-energy-alpha-LT}}
    \subfloat[Energy consumption, HT \label{fig:delay-alpha-0}]{\includegraphics[width=0.45\columnwidth]{image/algorithm/Final-energy-alpha-HT}}
  \caption{\textcolor{red}{Performance comparison between \gls{moroa} using different $\alpha$ and \gls{edca}, for a heterogeneous network} \label{fig:sumo-throughput-csma}}
\end{figure*}




In this section, we study the performance of \gls{moroa} in a network with heterogeneous stations. Moreover, we evaluate the effects of the $\alpha$ weight parameter on throughput and \textcolor{black}{energy consumption of both LT and HT stations}, for a variety of network densities. As \gls{etaroa} only supports stations with same \gls{mcs} and packet size, the results of \gls{moroa} are compared to \gls{edca} in this case. 

%Figure \ref{fig:throughput-alpha} shows that by increasing the value of $\alpha$, higher throughput can be achieved. Between $\alpha=0.25$ and $\alpha=0.5$, throughput increases $22$\%, $36$\%, $44$\% for $1200$, $1400$ and $1600$ stations respectively. While for $\alpha$ less than $0.25$ or larger than $0.5$, the throughput does not significantly change. For the small size networks, it shows there is no need to tune $\alpha$. Figure~\ref{fig:fairness-alpha-percent} illustrates the fairness with different $\alpha$, depicting the same conclusion.  With smaller values of $\alpha$, more fairness is achieved in terms of relative transmitted number of packets between LT and HT stations. Increasing $\alpha$ results in more focus on throughput, resulting in more airtime being provided to the HT stations. %The interesting finding from figure \ref{fig:sumo-alpha} is, although $\alpha$ is defined between 0 and 1, the actual range that affects the results is much smaller. More investigation are needed about tunning the $\alpha$, in order to achieve the required performance, it is considered as  our next step. 

%The comparison between \gls{moroa} and \gls{edca} is depicted in figure \ref{fig:sumo-throughput-csma}. With a small number of stations, \gls{moroa} and \gls{edca} have the same throughput. With more than 600 stations, the throughput of \gls{edca} stagnates to about 0.25~Mbps as the network becomes saturated. However, \gls{moroa} reduces the number of collisions, resulting in better scaling behavior and a higher saturation limit. As a result, for 1800 stations throughput of \gls{moroa} is about 44\% higher for $\alpha=0$ and  $\alpha=0.25$. If more focus is put on the throughput objective, with $\alpha=0.75$, \gls{moroa} more than doubles throughput compared to \gls{edca} in highly dense networks (i.e., 1800 stations), achieving a throughput increase of 108\%. As expected, \gls{edca} achieves high fairness among HT and LT stations due to its random access mechanism. For $\alpha \leq 0.25$ \gls{moroa} achieves a similar degree of fairness. However, for $\alpha = 0.75$ fairness of \gls{moroa} is significantly reduced, as it prioritizes HT stations to maximize overall throughput.

The comparison between \gls{moroa} and \gls{edca} is depicted in Figure~\ref{fig:sumo-throughput-csma}. With a small number of stations (i.e., 400 stations), \gls{moroa} and \gls{edca} have the same throughput. \textcolor{red}{For 600 stations, \gls{edca} still remains similar throughput as \gls{moroa} with $\alpha$ larger than 0.5.} While with more than 1000 stations, the throughput of \gls{edca} stagnates to about 0.047~Mbps and 0.21~Mbps for LT and HT stations respectively, as the network becomes saturated. However, for $\alpha = 1$ and $\alpha = 0.75$, \gls{moroa} reduces the number of collisions and results in better scaling behavior and a higher saturation limit. For example, with 1600 stations and $\alpha=0.75$, throughput of \gls{moroa} is about 38\% and 20\% higher for LT and HT stations, respectively. If more focus is put on the throughput objective, with $\alpha=1$, \gls{moroa} achieves about 65\% and 47\% higher throughput for LT and HT stations, respectively. In contrast, when $\alpha = 0$, \gls{moroa} obtains 31\% and 64\% less throughout for  LT and HT stations than \gls{edca} does.
In terms of energy consumption, \gls{moroa} always significantly outperforms \gls{edca}. For  1600 stations with $\alpha = 0$,  \gls{moroa} saves about 96\% and 91\% energy for LT and HT stations, respectively. 
%It should be noted that $\alpha = 0$ and $\alpha = 0.5$ result in the quite similar performance, it is because the range of objective $Q_t$ and $Q_e$ are different. As such, the $\alpha$ can be actually tuned between $0.5$ and $0.1$ to achieve the desired performance.

\begin{figure}[t]
  \centering
  \includegraphics[width=0.75\columnwidth]{image/algorithm/Final-collision-alpha-zoom-submission}
  \caption{\textcolor{red}{Collision comparison between MoROA using different $\alpha$ and EDCA/DCF, for a heterogeneous network.} \label{fig:collison}}
\end{figure}

\textcolor{red}{ The number of packets that collide with each other were counted during simulation, the results are depicted in Figure \ref{fig:collison}. It  clearly shows that, compare to \gls{edca}, \gls{moroa} significantly reduce the number of collisions. For 400 and 600 stations, the number of collision is 52845 and 60344 for \gls{edca}. As the number of station increases, the number of collision goes up dramatically, around 1122775 for 1600 stations. In contrast, the number of collision \gls{moroa} is at most 29142, 97.4\% less than \gls{edca}. As collision may cause the packets to be dropped, \gls{moroa} obtains higher performance (e.g., throughput, energy) than \gls{edca}. It should be noted that \gls{moroa} with $alpha = 0$  have less collision than \gls{edca}, but lower throughput than \gls{edca}. This is due to the fact \gls{moroa} with $alpha = 0$ tries to avoid the collision as much as possible, in order to minimize the energy consumption. As such, the airtime is not fully utilized, leading to lower throughput.}




%\begin{figure}[t]
%  \centering
%  \includegraphics[width=\columnwidth]{image/algorithm/Final-throughput-alpha-0}
%  \caption{Performance comparison between \gls{moroa}($\alpha=0.75$, $\alpha=0.25$, $\alpha=0.00$), and EDCA/DCF for heterogeneous network for LT and HT stations respectively. \label{fig:sumo-iteraton}}
%\end{figure}



% \begin{figure*}[t]
%     \subfloat[Ratio 25\%:75\%, throughput \label{fig:traffic-pattern-throughput}]{\includegraphics[width=0.5\textwidth]{image/algorithm/throughput-static-taroa-nc-aidre-025}}
%     \subfloat[Ratio 25\%:75\%, latency \label{fig:traffic-pattern-throughput}]{\includegraphics[width=0.5\textwidth]{image/algorithm/delay-static-taroa-nc-aidre-025}}
%  % \caption{Performance comparison between SUMO-E-TAROA and EDCA/DCF for mixed scenario with 25\% LT stations and 75\% stations. \label{fig:sumo-throughput}}
% %\end{figure*}

% %\begin{figure*}[t]
%     \subfloat[Ratio 50\%:50\%, throughput \label{fig:traffic-pattern-throughput}]{\includegraphics[width=0.5\textwidth]{image/algorithm/throughput-static-taroa-nc-aidre-050}}
%     \subfloat[Ratio 50\%:50\%, latency \label{fig:traffic-pattern-throughput}]{\includegraphics[width=0.5\textwidth]{image/algorithm/delay-static-taroa-nc-aidre-050}}
% %  \caption{Performance comparison between SUMO-E-TAROA and EDCA/DCF for mixed scenario with 50\% LT stations and 50\% stations. \label{fig:sumo-throughput}}
% %\end{figure*}






%\subsection{dynamic traffic pattern (optional)}



% \section{Conclusion and Future Work \label{sec:conclusion}}

In this paper, we present a novel solution for real-time \gls{raw} parameter optimization for IEEE~802.11ah, consisting of two contributions. First, we present a new \gls{raw} performance model, based on supervised surrogate modeling. 
%It can be easily trained on a limited set of labeled data samples, which can be obtained through simulation. 
\textcolor{red}{It can be easily trained on a limited set of labeled data samples, for a variety of performance metrics.}
Moreover, it is very fast to evaluate once trained, allowing it to be used for real-time \gls{raw} parameter optimization. The second contribution encompasses a \gls{raw} optimization algorithm called \gls{moroa}. 
%It uses the surrogate model to determine the optimal \gls{raw} configuration under a variety of network and traffic conditions. Moreover, it supports heterogeneous stations with different \gls{mcs} and average packet sizes.
\textcolor{red}{It uses surrogate models for the throughput and energy performance metrics to determine the optimal \gls{raw} configuration through multi-objective optimization under a variety of network and traffic conditions.} Moreover, it supports heterogeneous stations with different \gls{mcs} and average packet sizes.

%\textcolor{red}{TODO: List some of the main conclusions of the simulation experiments. Done}
The simulation results reveal three key points. First, the built surrogate model for \gls{raw} gets high accuracy relative to realistic simulation results. With a training set of $0.0074$\% of all possible data points, a relative error less than $7$\% for $90$\% of the randomly tested \gls{raw} configurations is achieved. 
%Second, by using the built surrogate model for \gls{raw}, \gls{moroa} achieves more stable throughput for both low- and high-density deployments as well as up to $32.5$\% lower latency, compared to the state of the art \gls{taroa} algorithm. 
\textcolor{red}{Second, by using the built surrogate model for \gls{raw} and multi-objective approach, \gls{moroa} is able to either improve throughput up to 5.3\%,  or improve energy energy efficiency by 21\% at only a minor 4.4\% throughput loss, compared to the state of the art \gls{etaroa} algorithm. Most importantly, \gls{moroa} supports traffic-aware \gls{raw} optimization for heterogeneous scenarios with variable \glspl{mcs} and packets size. Its objective weight allows to attain either a 65\% throughput increase, or a 96\% energy saving, or a weighted solution in between.
%achieving up to 65\% higher throughput and up to 96\% less energy consumption than \gls{edca} in dense networks. 
}


In future work, we aim to further extend the surrogate modeling approach to support stations with different \gls{mcs} and average packet sizes not only across multiple groups, but also within a single \gls{raw} group. This would further increase the flexibility of the algorithm in finding an optimal \gls{raw} configuration. Moreover, more advanced clustering metrics will be evaluated and compared for use with \gls{moroa}. Finally, we will investigate \gls{aid} reassignment methods to allow \gls{moroa} to satisfy the sequential \gls{aid} requirement of the 802.11ah standard, and take the co-existence of multiple \gls{ap}s into account.


% \section*{Acknowledgement}
% Part of this research was funded by the Flemish FWO SBO S004017N IDEAL-IoT (Intelligent DEnse And Long range IoT networks) project. Serena Santi is funded by the Fund For Scientific Research (FWO) Flanders under grant number 1S82118N.


% \bibliographystyle{IEEEtran}
% \bibliography{references.bib}





