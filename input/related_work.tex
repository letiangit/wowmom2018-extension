\section{Related Work \label{sec:related_work}}

Since the \gls{raw} feature was proposed, several studies have been conducted on the evaluation of \gls{raw} performance. Raeesi \textit{et al.} demonstrate that the \gls{raw} mechanism can provide substantial improvements in terms of throughput, delay and energy consumption, in particular in highly-loaded dense network scenarios \cite{Raeesi2014a}. In our own previous work~\cite{WoWMoM2016}, we further evaluated the optimal \gls{raw} station grouping configuration under a variety of traffic conditions, such as traffic load, number of stations and \gls{raw} group duration on the optimal number of \gls{raw} groups. These works prove the strong correlation between network and traffic conditions on one hand, and the optimal \gls{raw} configuration on the other. This supports the hypothesis that there is a need for real-time \gls{raw} parameter optimization. 

%\subsection{\gls{raw} performance models}

To determine the optimal \gls{raw} parameters, several analytical models have been proposed to calculate \gls{raw} performance under specific network and traffic conditions. These models make use of different techniques, such as probability theory~\cite{Wang2015}, Markov chains~\cite{Khorov2015b,Zheng2014}, and maximum likelihood estimation~\cite{Park2014b}. Early works assume the network is operating under saturated state, where each station always has packets to send \cite{Zheng2014, Park2014b}. This is not a very realistic assumption for \gls{iot} and \gls{mtc}~\cite{Khorov2015b}. The model proposed by Zheng \textit{et al.} considers both cross and non-cross slot boundary traffic, and is able to calculate the throughput with any given number of stations and \gls{raw} duration~\cite{Zheng2014}. \textcolor{red}{A more accurate mathematical model was recently developed by Lyakhov \textit{et al.} \cite{Evgeny2018}, by taking into account the non-steady state of the back-off function at the beginning of the \gls{raw} period.} Park \textit{et al.} determine the \gls{raw} group duration for a certain number of stations to get maximized successful transmission probability~\cite{Park2014b}. In contrast, more recent works assume each station sends one packet per \gls{raw} slot interval~\cite{Khorov2015b, Wang2015, Bel2014}. Khorov \textit{et al.} built a model to analyze the successful packet transmission probability under a given \gls{raw} group duration~\cite{Khorov2015b}. The model of Wang \textit{et al.} focuses on energy consumption~\cite{Wang2015}. \textcolor{red}{Chang \textit{et al.} took a step further, supporting more diverse traffic demands \cite{Chang2018}. They use the results of two extreme cases (i.e.,  infinite traffic and one packet sent per \gls{raw} slot) to extrapolate a regression-based analytical model that can accurately fit the contention success probability of any traffic patterns. However, the regression model does not take the finite length of the RAW slot into account.}

All existing analytical models share two main shortcomings. First, they are computationally hard. This makes it unfeasible to execute them in real-time on actual \gls{ap} hardware, where at most a few milliseconds are available at the start of the beacon interval to calculate a new \gls{raw} configuration. More importantly, they assume ideal channel conditions, without communication errors, delays or capture effects, \textcolor{red}{ or did not takes into account the important peculiarity of the RAW mechanism.}
The combination of these factors make such models useful only from a theoretical point of view, to analyze the effectiveness of \gls{raw} under a variety of conditions. However, they cannot be used for real-time station grouping under dynamic and realistic traffic conditions. Our proposed surrogate modeling approach aims to address both of these issues.

%\subsection{\gls{raw} optimization algorithms}

In addition to modeling \gls{raw} performance, it is necessary to use this information in real-time, in order to optimize \gls{raw} parameters in an actual network. Current solutions are mainly based on set partitioning. These \gls{raw} optimization algorithms assume the number of \gls{raw} slots and groups is predetermined, and decide how to partition the associated stations among them, according to some metric. Their simplicity makes it computationally feasible to deploy them in real networks. Several algorithms utilize \gls{raw} to mitigate hidden node collisions by splitting mutually hidden nodes into orthogonal groups~\cite{Yoon2016,Damayanti2016}%{Yoon2016,Dong2016,Damayanti2016}
. Chang \textit{et al.} proposed a set partitioning algorithm that assumes the (static) traffic demand of each station is known by the \gls{ap} and load balances them across groups~\cite{Chang2015}. Other existing algorithms focus on simple partitioning metrics, such as fully random~\cite{Ogawa2013} or based on the back-off timer value~\cite{Qutab-Ud-Din2015}, which in reality is not known to the \gls{ap}. Such set partitioning algorithms have several shortcomings. First, high channel contention exists in dense sensor network even without the presence of hidden nodes. Reducing hidden nodes can mitigate collisions to some extent, but is not sufficient. Second, they expect all information, such as the exact traffic intensity of each station, to be readily available at the \gls{ap} side, which in reality is not the case. Third, they assume that the number of groups and slots as well as their duration are predefined, and only the partitioning of stations among them needs to be solved. The number of groups and their duration, however, significantly influence \gls{raw} optimality~\cite{WoWMoM2016}. Finally, none of the presented algorithms take into account traffic dynamics. In a real network, the upstream traffic intensity of stations may change over time for a variety of reasons, and the algorithm should therefore adapt to these changes.

% Recently, we proposed the \gls{taroa} \cite{Sensor2017,Sensys2017}. It adapts the optimal \gls{raw} parameters in real-time by estimating the current traffic conditions, based solely on information available at the \gls{ap}. However, it still has two shortcomings that can be addressed. 
% \textcolor{red}{First, it derives the optimal number of stations to assign to a group based on saturated state simulation results on throughput.} Second, it only supports homogeneous stations (i.e., all stations use the same MCS and average packet size). 

\textcolor{red}{
We had proposed the \gls{taroa} \cite{Sensor2017,Sensys2017}. It adapts the optimal \gls{raw} parameters in real-time by estimating the current traffic conditions, based solely on information available at the \gls{ap}. However, it derives the optimal number of stations to assign to a group based on saturated state simulation results on throughput. Recently, \cite{Tian2018} was proposed. Different from \cite{Sensor2017,Sensys2017},  it uses generic and flexible surrogate models on throughput to determine the optimal \gls{raw} configurations, resulting in significant performance improvements. Moreover, it supports a wide range of traffic conditions and heterogeneous stations. However, \cite{Tian2018} focus on throughput. Energy efficiency is also a major concern in \gls{iot} networks, as sensors are battery powered and they are supposed to work for years. Therefore, in this paper we present an improved version, called \textcolor{yellow}{xx}, it generates surrogate models on both throughput and energy consumption, and use RAW configuration by optimizing both throughput and energy simultaneously. }
%This results in significant performance improvements, especially under non-saturated conditions, which are prevalent in \gls{iot} and \gls{mtc} scenarios. 






%use the simulation results (since the analytic model \cite{Zheng2014} does not support capture effect) of saturated state to derive the optimal number of stations $\sigma^r_\textit{opt}$, while in most case network is not saturated. Second, it can only support homogeneous station, i.e., all station have the same MCS and packet size.




% In contrast to other state-of-the-art algorithms it is capable of adjusting its \gls{raw} configuration in real-time, in face of station and traffic dynamics. Furthermore, by exploiting the “more data” header field \textcolor{white}{and cross slot boundary} feature, a more accurate traffic estimation technique for IEEE 802.11ah sensor stations was proposed, which is integrated into an enhanced version of the Traffic-Adaptive \gls{raw} Optimization Algorithm, referred to as E-TAROA \cite{Sensys2017}. With more accurate traffic estimation in very dense networks with thousands of sensor stations, E-TAROA results in a significantly more optimal \gls{raw} configuration. Specifically, E-TAROA converges significantly faster and achieves up to 23\% higher throughput and 77\% lower latency than the original TAROA algorithm under high traffic loads.


