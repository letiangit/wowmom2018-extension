\section{Introduction}

The recently released long-range and low-power Wi-Fi standard IEEE~802.11ah proposes a novel channel access method, referred to as \gls{raw}. It is a flexible hybrid method, highly suited to provide scalable connectivity to both sparsely and densely deployed low-power devices. \gls{raw} is based on station grouping and attempts to reduce contention and collisions in highly dense deployments by dividing stations into groups and allowing channel access to one group at a time. Consequently, IEEE~802.11ah allows up to 8192 stations to connect to a single \gls{ap}. 

Figure~\ref{fig:RAW} schematically depicts how \gls{raw} works. Specifically, the channel airtime is split into several intervals, some of which are assigned to RAW groups, while others are shared and can be accessed by all stations using the traditional 802.11 \gls{edca}, which rely on \gls{csma} channel access. At fixed intervals a beacon frame is transmitted, carrying a \gls{rps} information element. The \gls{rps} specifies the stations belonging to each group using the start and end \gls{aid}, the group start time, and duration. Moreover, each RAW group consists of one or more equal-duration slots, among which the stations assigned to the RAW group are evenly split (using round robin assignment). The \gls{rps} information element also contains the number of slots, slot format and slot duration count sub-fields, which jointly determine the RAW slot duration.
\textcolor{black}{
Moreover, \gls{raw} supports two types of transmissions indicated by the \gls{csb} sub-field of the \gls{rps} information element. When the \gls{csb} sub-field is set to false, stations should not start a transmission if the remaining time in the current \gls{raw} slot is not enough to complete it. This remaining time is termed as holding period. Alternatively, stations are allowed to continue their ongoing transmissions even after the end of the current RAW slot when the \gls{csb} sub-field is set to true.}
For a more in-depth description of \gls{raw}, the reader is referred to existing literature~\cite{Khorov2015a,Sensor2017}.

\begin{figure}[t]
  \centering
  \includegraphics[width=0.8\columnwidth]{image/raw.pdf}
  \caption{Schematic representation of the \gls{raw} mechanism.\label{fig:RAW}}
\end{figure}

The 802.11ah standard, however, does not specify how to configure the actual \gls{raw} grouping parameters. Additionally, previous research has shown that the optimal \gls{raw} configuration depends on a variety of network-related parameters, such as the number of stations, traffic patterns, and network load~\cite{WoWMoM2016}. Incorrect configuration severely impacts throughput, latency and energy efficiency. As such, there is a need for \gls{raw} optimization algorithms that collect network-related information, and at the start of each beacon interval adapt the \gls{raw} configuration based on the current network conditions. Such an algorithm should be able to calculate a new \gls{raw} configuration in real-time (i.e., at most a few milliseconds), as it needs to use network-related information obtained from the previous beacon interval and calculate a solution before the new \gls{rps} information element is broadcast. Moreover, in order to select the optimal \gls{raw} parameters, it should be able to predict \gls{raw} performance for a given set of parameters under specific network and traffic conditions. This is achieved using some sort \textcolor{red}{of} model of the environment, which takes as input network conditions and a \gls{raw} configuration, and generates as output one or more performance metrics (e.g., throughput or energy consumption).

In the past, several analytic models have been proposed to predict \gls{raw} performance~\cite{Khorov2015b,Wang2015}. However, \st{ such models are too computationally hard to be used in real-time,}  \textcolor{red}{such models rely on simplifications and unrealistic assumptions (e.g., no capture effect, no hidden nodes, homogeneous stations, saturated or static traffic).} As a first contribution, we present an alternative solution to \gls{raw} performance modeling, based on surrogate modeling and trained using realistic simulation results. A surrogate model is based on supervised learning (e.g., Kriging, or neural networks), but can be accurately trained with very few labeled sample data points~\cite{SUMOtoolbox2010} . This is important, as a \gls{raw} configuration depends on many input variables that can take a wide range of values. \st{Moreover, once trained, evaluating the model is equivalent to a constant-time table look-up, which can be easily executed in real-time.}
As there is no IEEE 802.11ah hardware available, training is done using the realistic IEEE~802.11ah ns-3 model~\cite{WNS32016}. It implements the capture effect, hidden stations, as well as heterogeneous stations and  traffic loads.
%By using realistic simulation results, obtained from the IEEE~802.11ah implementation in ns-3~\cite{WNS32016}, the model takes into account the capture effect, as well as heterogeneous stations and different traffic loads.

%As a second contribution, we propose the \gls{moroa}. It clusters stations into groups based on their traffic characteristics and determines the optimal \gls{raw} configuration by solving a non-linear constrained optimization problem. In this problem, the trained surrogate model is used to maximize throughput and fairness in terms of packets delivery ratio. 
As a second contribution, we propose the \gls{moroa}. It clusters stations into groups based on their traffic characteristics and determines the optimal \gls{raw} configuration by solving a non-linear constrained optimization problem. The trained surrogate model is used to maximize throughput and energy efficiency simultaneously. In contrast to existing algorithms, \gls{moroa} supports multi-objective optimization of both throughput and energy, dynamic and heterogeneous traffic. \textcolor{red}{Moreover, by assigning stations into homogeneous groups, \gls{moroa} supports heterogeneous stations with different \glspl{mcs} and packet sizes~\cite{Sensor2017,Sensys2017}.} \st{as well as heterogeneous stations with different MCSs and packet sizes}.

The remainder of this paper is structured as follows. Section~\ref{sec:related_work} surveys related work in terms of \gls{raw} performance modeling and optimization algorithms and compares them to our contributions. Section~\ref{sec:SUMO} details the methodology used to define and train the surrogate model. \gls{moroa} is described in Section~\ref{sec:algorithm}. Section~\ref{sec:evaluation} evaluates the accuracy of our presented model, comparing it to %\textcolor{black}{state of the art analytical models and} 
simulation results. Moreover, performance of \gls{moroa} is evaluated and compared to state of the art \gls{raw} optimization algorithms, as well as the traditional \gls{edca} function of IEEE~802.11. Finally, Section~\ref{sec:conclusion} offers conclusions and a short overview of future work.


%Several algorithms have been proposed to determine suitable RAW parameters. For sensor network traffic with either 1 packet per station or under saturation, some analytical models were proposed. These models are based on different techniques, such as probability theory~\cite{Wang2015,Raeesi2014a}, Markov chains~\cite{Khorov2015b,Zheng2014}, multi-objective game theory~\cite{Bel2014}, and maximum likelihood estimation~\cite{Park2014b}. However, these models are computationally hard, which makes it infeasible to execute them in real-time on actual AP hardware. As an alternative that is computationally feasible and deployable, several partitioning algorithms were proposed. They partition the stations into different RAW slots based on different metrics, such as arbitration inter-frame space number (AIFSN) value~\cite{Ogawa2013}, and station traffic load ~\cite{Chang2015}. However, this information is not known to the AP in reality, also making them infeasible to implement. Recently, we proposed a real-time station grouping algorithm, named TAROA, by estimating the traffic conditions of each station with information only available at the AP~\cite{Sensor2017}. In contrast to other state-of-the-art algorithms it is capable of adjusting its RAW configuration in real-time, in face of station and traffic dynamics. 