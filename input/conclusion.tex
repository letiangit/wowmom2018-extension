\section{Conclusion and Future Work \label{sec:conclusion}}

In this paper, we present a novel solution for real-time \gls{raw} parameter optimization for IEEE~802.11ah, consisting of two contributions. First, we present a new \gls{raw} performance model, based on supervised surrogate modeling. 
%It can be easily trained on a limited set of labeled data samples, which can be obtained through simulation. 
\textcolor{red}{It can be easily trained on a limited set of labeled data samples, for a variety of performance metrics.}
Moreover, it is very fast to evaluate once trained, allowing it to be used for real-time \gls{raw} parameter optimization. The second contribution encompasses a \gls{raw} optimization algorithm called \gls{moroa}. 
%It uses the surrogate model to determine the optimal \gls{raw} configuration under a variety of network and traffic conditions. Moreover, it supports heterogeneous stations with different \gls{mcs} and average packet sizes.
\textcolor{red}{It uses surrogate models for different performance metrics to determine the optimal \gls{raw} configuration under a variety of network and traffic conditions.} Moreover, it supports heterogeneous stations with different \gls{mcs} and average packet sizes.

%\textcolor{red}{TODO: List some of the main conclusions of the simulation experiments. Done}
The simulation results reveal three key points. First, the built surrogate model for \gls{raw} gets high accuracy relative to realistic simulation results. With a training set of $0.0074$\% of all possible data points, a relative error less than $7$\% for $90$\% of the randomly tested \gls{raw} configurations is achieved. 
%Second, by using the built surrogate model for \gls{raw}, \gls{moroa} achieves more stable throughput for both low- and high-density deployments as well as up to $32.5$\% lower latency, compared to the state of the art \gls{taroa} algorithm. 
\textcolor{red}{Second, by using the built surrogate model for \gls{raw}, \gls{moroa} achieves about 4.4\% higher as well as up to $21$\% lower energy consumption, compared to the state of the art \gls{etaroa} algorithm. Most importantly, \gls{moroa} supports traffic-aware \gls{raw} optimization for heterogeneous scenarios with variable \glspl{mcs} and packets size, achieving up to 65\% higher throughput and up to 96\% less energy consumption than \gls{edca} in dense networks. }


In future work, we aim to further extend the surrogate modeling approach to support stations with different \gls{mcs} and average packet sizes not only across multiple groups, but also within a single \gls{raw} group. This would further increase the flexibility of the algorithm in finding an optimal \gls{raw} configuration. Moreover, more advanced clustering metrics will be evaluated and compared for use with \gls{moroa}. Finally, we will investigate \gls{aid} reassignment methods to allow \gls{moroa} to satisfy the sequential \gls{aid} requirement of the 802.11ah standard, and take the co-existence of multiple \gls{ap}s into account.