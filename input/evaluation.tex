\section{Performance Evaluation and Discussion \label{sec:evaluation}}

This section presents the evaluation results of the \gls{raw} performance of surrogate model and the \gls{moroa} \gls{raw} optimization algorithm. First, the simulation setup is discussed. Second, the accuracy of the surrogate model is compared to simulation results. Finally, \gls{moroa} is evaluated and compared to state of the art \gls{raw} algorithms, as well as the traditional \gls{edca} method of IEEE~802.11.

\begin{figure}[t]
  \centering
  \includegraphics[width=0.75\columnwidth]{image/throughput_cv}
  \caption{\textcolor{black}{Cross validation score of the surrogate models as a function of the number of training \textcolor{red}{sample data points}.} \label{fig:sumo-iteraton}}
\end{figure}

\subsection{Simulation setup}


% All evaluations are performed using our previously developed 802.11ah ns-3 module~\cite{WNS32016}, based on ns-3 version 3.23. We consider the same IoT scenario as described in section \ref{sec:algorithm},  the default PHY and MAC layer parameters used in our simulation are shown in Table~\ref{tab:parameters}. Concretely,  we consider two types of stations (i.e., HT and LT station as defined in section \ref{sec:SUMO}) co-exists in the one network, each station randomly choose a value X (second) from the range [1, 10] as its transmission interval. We use $\eta$to tune the percentage of the number of each type of station in the whole network, $\eta$, and ($100-\eta$) stations belongs to HT and LT respectively, $\eta=\left[0, 0.25, 0.50, 0.75, 1\right]$~\% are evaluated in the experiment. 
   

% RAW performance is evaluated in terms of two metrics: throughput, fairness, and latency. Throughput is calculated as the average number of successfully received payload bytes by the AP per second. Fairness among HT and LT stations are evaluated using Jain's fairness index \cite{jain1984quantitative}. Latency is defined as the average time between a packet entering the transmit queue of the station and being received by the AP. Each simulation runs 600~s . 


All evaluations are performed using our previously developed IEEE~802.11ah ns-3 module~\cite{WNS32016}, based on ns-3 version~3.23. We consider the same \gls{iot} scenario as described in Section~\ref{sec:algorithm}. The same default PHY and MAC layer parameters used as shown in Table~\ref{tab:ns3 parameters}. We consider both homogeneous and heterogeneous scenarios. Homogeneous scenarios are used to validate the surrogate model and compare to \gls{etaroa}~\cite{Sensys2017}. In heterogeneous scenarios, half of the stations use the high-throughput (HT) settings and half of them use the low-throughput settings (LT) listed in Table~\ref{tab:ns3 parameters}. The data transmission interval of each station is selected uniformly at random from the interval $\left[1, 10\right]$~seconds.

%RAW performance is evaluated in terms of three metrics: throughput, fairness, and latency. Throughput is calculated as the average number of successfully received payload bytes by the \gls{ap} per second. Fairness between HT and LT stations in terms of packet delivery ratio is evaluated using Jain's fairness index. Latency is defined as the average time between a packet entering the transmit queue of the station and being received by the \gls{ap}. 

\textcolor{black}{\gls{raw} performance is evaluated in terms of two metrics: throughput and energy. Throughput is calculated as the average number of successfully received application payload bits by the \gls{ap} per second. Energy is considered as the total energy consumption averaged over the number of successfully received packets.} Each simulation runs for 600~seconds, and all results are averaged over 10 iterations, with the variability of results over these iterations quantified using the standard deviation (SD).



% Run original E-TAROA (one station per slot) until the estimation is stable, then switch to E-TAROA(SUMO). \textcolor{black}{when should the switch happen?}
% \\
% How to switch:\\
% 1. reassign AID based on the transmission interval. Cluster stations based on their AIDs, resulting in:\\
% Interval 1, $n_1$ stations \\
% Interval 2, $n_2$ stations \\
% ... ...\\
% Interval k, $n_k$ stations \\

\subsection{SUMO model validation}




%        \includegraphics[clip, trim=0.5cm 11cm 0.5cm 11cm, width=1.00\textwidth]{gfx/BI-yourfile.pdf}

\begin{figure}[t]
  \centering
    \includegraphics[width=0.75\columnwidth]{image/model_error}
  \caption{\textcolor{black}{Performance comparison between the surrogate model and simulation results for 5000 random test data points. %, sorted in ascending order in terms of estimation error
  }\label{fig:sumo-test-data}}
\end{figure}

In this section we evaluate the training convergence of the surrogate model, as well as its accuracy compared to simulation results. Figure~\ref{fig:sumo-iteraton} plots the model's cross validation score as a function of the number of training samples used. The cross validation score provides a measure for the accuracy of the resulting model \textcolor{red}{on predication of \gls{raw} performance using the IEEE~802.11ah ns-3 simulator}. A consistently low score signifies that the training process has converged. 
%Based on this graph, we can conclude that for both the model with high-throughput (HT) and low-throughput (LT) stations, convergence occurs after around $1700$ training samples have been used.
The results show, for the throughput models of both HT and LT stations, that the convergence occurs after around $1700$ training samples have been used. This comes down to about $0.0074$\% of all data points in the input space (i.e., $2.3 \times 10^7$), and about $6.8$\% of the reduced data space (i.e., $25047$) from which samples were drawn during training. The convergence of energy models happens after 1900 and 2500 training samples have been used for energy models of  HT and LT stations, respectively. The training of the HT throughput model and LT energy model stopped after 2370 and 2400 \st{iterations}  \textcolor{red}{training samples} respectively, as it satisfied the cross-validation stop conditions, i.e., 10 consecutive cross-validation scores (2 digits of precision) remain the same. The training of the other models stopped after the maximum number of 2500 \st{iterations} \textcolor{red}{training samples}  , having achieved 15 consecutive cross-validation scores between $0.11$ and $0.12$ for the LT throughput model, and 9 consecutive cross-validation scores between $0.054$ and $0.058$ for the HT energy model.  The cross-validation scores decrease as more training samples are used, is due to the sampling strategy FLOLA-Voronoi. It selects points from non-linear regions and the sparsely sampled regions, making the model more and more accurate. It should be noted that the cross-validation scores fluctuate sometimes. This is because new training sample points are selected from an untouched region, and the surrogate model is  recalculated to consider this change.

%In order to ensure no over-fitting occurred, the surrogate model also provides accurate results for data points outside the reduced input space (cf. Table~\ref{tab:sumo parameters}), Figure~\ref{fig:sumo-test-data} plots the absolute error (in terms of throughput) of the surrogate model compared to simulation results. In total, 2070 random data points were generated from all $2.3 \times 10^7$ possible points. The figure also plots the actual simulated throughput of each point, to characterize the significance of the error. For the HT model, the absolute throughput estimation error stays below 0.02~Mbps for the first 2000 points (i.e., $96.6$\% of all data points). The absolute error only goes above $0.1$~Mbps for 3 data points (i.e., $0.14$\%). Similarly, for the LT model, the throughput estimation error stays below $0.004$~Mbps for the first 2000 data points, and only grows above $0.01$~Mbps for the worst 11 data points. In terms of the relative error (i.e., ratio between absolute throughput error and simulation results, not depicted) of the surrogate model compared to simulations, 95\% of all data points have an error below 6.6\% and 5.4\% for the HT and LT model respectively. For both models, 98\% of the data points have a relative error below 10\%. These results validate the ability of the surrogate model to estimate \gls{raw} performance accurately, even for points outside of the reduced input data space used for initial training.

\textcolor{black}{In order to ensure no over-fitting occurred, we evaluated if the surrogate model also provides accurate results for data points outside the reduced input space (cf. Table~\ref{tab:sumo parameters}). Figure~\ref{fig:sumo-test-data} plots the \gls{cdf} of the relative error (i.e., ratio between absolute error and simulation results). In total, 5000 random data points were generated from all $2.3 \times 10^7$ possible points. It shows around 95\% and 98\% of the test data points have a relative error less than 10\% for the throughput and energy models for LT stations, respectively. Similar results are obtained for the HT stations. 90\% of the test data points have a relative error less than 7\% for both the throughput and energy models of LT stations. For the throughput and energy models of HT stations, this is respectively 7\% and 5\%. These results validate the ability of the surrogate model to estimate \gls{raw} performance accurately, even for points outside of the reduced input data space used for initial training.}


\subsection{\gls{raw} configuration analysis}

% \begin{figure}[t]
%     \centering
%     \subfloat[LT, energy per packet \label{fig:traffic-pattern-throughput}]{\includegraphics[width=0.35\textwidth]{image/LT-pareto-front-energyPerPacket}}\\
%     \subfloat[Energy consumption per packet \label{fig:traffic-pattern-throughput}]{\includegraphics[width=0.35\textwidth]{image/HT-pareto-front-energyPerPacket}}
%   \caption{\gls{raw} configurations for top 10\% highest energy efficiency. \label{fig:ana-energy}}
% \end{figure}

\textcolor{black}{In this section, based on the built model, we analyze the impact of different \gls{raw} configurations on throughout and energy consumption, and derive the optimal \gls{raw} configuration. For each number of stations, we first find the highest output $t_{max}$ from the built model $\mathcal{F}_t\left(\cdot\right)$, and find $e_{min}$ from the model $\mathcal{F}_e^p \left(\cdot\right)$, which is calculated as follows:
\begin{equation} \label{eq:objectiveF}
\mathcal{F}_e^p (n_r, d_r, s_r) =     \frac{ \mathcal{F}_e (n_r, d_r, s_r) \times l_r }{\mathcal{F}_t\left(n_r, d_r, s_r\right) \times \textcolor{red}{T_c} }  
\end{equation}
The model $\mathcal{F}_e^p (n_i, d_i, s_i)$ predicts the energy consumption per packet for a \gls{raw} group $r$ with duration $d_r$, consisting of $s_r$ slots, and with $n_r$ stations assigned to it. Subsequently, we select the \gls{raw} configurations that deviate at most 10\% from the optimal throughput $t_{max}$ and the optimal energy consumption $e_{min}$.
%that lead to near optimal performance, i.e., throughput higher than 90\% of $t_{max}$, or energy consumption lower than 110\% of $e_{min}$.
}

\begin{figure*}[t]
    \centering
    \subfloat[Throughput \label{fig:traffic-pattern-throughput}]{\includegraphics[width=0.45\columnwidth]{image/LT-pareto-front-throughput}}
    \subfloat[Energy consumption per packet \label{fig:traffic-pattern-energy}]{\includegraphics[width=0.45\columnwidth]{image/LT-pareto-front-energyPerPacket}}
  \caption{\gls{raw} configurations that deviate at most 10\% from the optimal throughput or energy consumption. \label{fig:ana-throughput}}
\end{figure*}

\begin{figure*}[!h]
    \centering
    \subfloat[100 stations \label{fig:pareto-LT}]{\includegraphics[width=0.45\columnwidth]{image/pareto-front101}}
    \subfloat[400 stations \label{fig:pareto-HT}]{\includegraphics[width=0.45\columnwidth]{image/pareto-front401}}
    \caption{Pareto front of throughput and energy per station, pareto front points are highlighted in red. \label{fig:sumo-pareto}}
\end{figure*}

\textcolor{black}{Figure~\ref{fig:ana-throughput}  depicts such \gls{raw} configurations for 100, 200 and 300 LT stations, respectively. As the figures show, if the number of stations is relatively small (i.e., 100), many different \gls{raw} configurations lead to near optimal throughput. This is because channel contention is less fierce in such a scenario, and the \gls{raw} configuration thus has less effect on throughput. Even without \gls{raw} (i.e., one \gls{raw} slot), quite high performance can already be obtained, while a larger number of \gls{raw} slots is definitely needed when the number of contending stations increases to 200 or 300). It should be noted that the near optimal number of \gls{raw} slots for throughput is discontinuously distributed. For example, for \gls{raw} duration 204.8 ms and 200 stations, a number of slots in the intervals 9--16, 21--26 and 40--50 achieve the highest throughput. This is because cross-slot boundary transmissions were disabled, which leads to varying amounts of wasted airtime (i.e., holding time) \cite{Evgeny2018}. Different from throughput, there are fewer near optimal \gls{raw} configurations for energy.}



\textcolor{black}{There exist some \gls{raw} configurations that can maximize throughput and minimize energy simultaneously, as depicted in Figure~\ref{fig:sumo-pareto}. This figure illustrates the Pareto front of throughput and energy consumption per packet for 100 and 400 LT stations. As shown, the 400 stations have only one Pareto front solution, i.e., the objective of maximizing throughput and minimizing energy consumption can be achieved simultaneously. While for 100 stations there are several Pareto front solutions. Among them, a solution can be chosen based on the relative importance of each of the two objectives, using a multi-objective \gls{raw} optimization algorithm such as \gls{moroa}.}

%For a certain slot number $s_r$ , once a high throughput can be achieved, a similar throughput can also be obtained with slot number $s_r^{'}$ larger than $s_r$. The reason is, $s_r^{'}$ number leads to less number of stations assigned to one \gls{raw} slot, therefore mitigate the channel contention, and maintain the throughput, or increase the throughput if there are more packets need to be sent. However, there are several exceptions. First of all, if the number of slots is too big, it may cause the \gls{raw} slot duration too short to transmit a single packet. Second,  \todo{the effect of non-cross slot boundary}. While for cross slot boundary, the performance fluctuation does not occur  \cite{WoWMoM2016} and \cite{Zheng2014}. Second, it can be observed that the normalized throughput increases in a fluctuating way as the duration of the RAW slot increases. The reason is that a varying number of mini-slots can be wasted due to the RAW slot handover between groups.


% \begin{enumerate}
% \item There is a set of \gls{raw} slot numbers that can achieve high performance. 
% \item For shorter \gls{raw} duration, there is less options on the \gls{raw} slot number.
% \todo{affected by non cross slot boundary}
% \item For a certain number of stations, there is a requirement on the minimal \gls{raw} duration, denoted as \gls{d_r^{min}}. Any \gls{raw} duration larger than \gls{d_r^{min}}, can leads to higher throughput, with more options on the \gls{raw} slot number.
% \todo{affected by non cross slot boundary}
% \end{enumerate}

% \textbf{findings for energy from simulation results}
% \begin{enumerate}
% \item Energy efficiency increase as \gls{raw} duration increase.
% \item Energy efficiency increase as \gls{raw} duration increase.
% \end{enumerate}

% \textbf{findings for total energy from \ref{fig:sumo-energy}}
% \begin{enumerate}
% \item Large \gls{raw} duration has high energy efficiency .
% \item Requirement on the minimal \gls{raw} duration.
% \item Requirement on the minimal \gls{raw} \gls{raw} slot number for relative small \gls{raw} duration. Since too few slot number results in high contention.
% \item No Requirement on the minimal \gls{raw} \gls{raw} slot number for large \gls{raw} duration. because that the \gls{raw} slot duration is big enough.
% \item Number of good \gls{raw} performance configuration decrease as station number increase.
% \end{enumerate}


% \textbf{findings for energy on per packet, similar to throughput}


% \textbf{findings for throughput, HT}
% \begin{enumerate}
%     \item 
% \end{enumerate}


% \textbf{summary of the figures, when there is not airtime limit}
% \begin{enumerate}
%     \item when there is not airtime limit \\
%     The packet deliver ratio and energy efficiency overlap, there exists some \gls{raw} configurations that can maximize both objectives simultaneously.  Except (findings from 2070 simulations)
%     \begin{enumerate}
%         \item exception, HT, 204800 us and slot = 1, always got good throughput
%         \item exception, HT, 204800 us and slot = 1, always got good energy from 61 to 336 stations, then slot = 50 get good energy efficiency. 336 stations is a break point, where the achievable throughput start to decrease.
%          \item exception, LT, 204800 us and slot = 1, always got energy until 216 stations. Then 50 stations get good energy efficiency.
%          \item exception, LT, 204800 us and slot = 1, always got throughput until 216 stations. Then 50 stations get good throughput.

         

%     \end{enumerate}
%     \item when airtime is an valuable resource,  i.e., throughput is considered with packet deliver ratio higher than 99\%  \\
%     there is a tradeoff.
% \end{enumerate}













\subsection{Homogeneous stations}

\begin{figure*}[t]
    \centering
    \subfloat[LT, Throughput \label{fig:traffic-pattern-throughput}]{\includegraphics[width=0.45\columnwidth]
    {image/algorithm/throughput-aidaddress-alpha-1-0}} 
    \subfloat[LT, Energy consumption \label{fig:traffic-pattern-throughput}]{\includegraphics[width=0.45\columnwidth]
    {image/algorithm/Energy_aidaddre_ncr-alpha-1-0}} \\
    \subfloat[HT, Throughput \label{fig:traffic-pattern-throughput}]{\includegraphics[width=0.45\columnwidth]
    {image/algorithm/throughput-homo-nohidden-etaroa-aidaddress-alpha-1-0-HT}} 
    \subfloat[HT, Energy consumption \label{fig:traffic-pattern-throughput}]{\includegraphics[width=0.45\columnwidth]
    {image/algorithm/EnergyHomo_etraoa_aidaddre_ncr-alpha-1-0-HT}}
  \caption{\textcolor{red}{Performance comparison between \gls{moroa} and \gls{etaroa} using $\alpha=0, 1$  for the LT and HT scenarios with different traffic loads and station counts} \label{fig:sumo-home}
  }
\end{figure*}

% \begin{figure}[t]
%   \centering
%     \includegraphics[width=0.75\columnwidth]{image/withalpha}
%   \caption{\textcolor{black}{The performance of \gls{moroa} as a function of different $\alpha$
% }\label{fig:homo-alpha}}
% \end{figure}

\begin{figure*}
    \centering
    \subfloat[LT, Throughput \label{fig:homo-alpha-throughput}]{\includegraphics[width=0.45\columnwidth]
    {image/withalpha-throughput}} 
    \subfloat[LT, Energy consumption \label{fig:homo-alpha-energy}]{\includegraphics[width=0.45\columnwidth]
    {image/withalpha-energy}} \\
    \subfloat[HT, Throughput \label{fig:homo-alpha-throughput}]{\includegraphics[width=0.45\columnwidth]
    {image/algorithm/withalpha-throughput-review}} 
    \subfloat[LT, Energy consumption \label{fig:homo-alpha-energy}]{\includegraphics[width=0.45\columnwidth]
    {image/algorithm/withalpha-energy-review}} 
  \caption{\textcolor{red}{Performance of \gls{moroa} as a function of different $\alpha$} \label{fig:homo-alpha}
  }
\end{figure*}







%Compare original E-TAROA (one station per slot) and revised E-TAROA (using SUMO modeling) in terms of throughput, latency and energy consumption. 
%
%In this section we evaluate the performance of \gls{moroa} for homogeneous stations (in terms of \gls{mcs} and packet size) for a variety of traffic loads and station counts. Three different total traffic loads are simulated for the LT scenario, i.e., $T$ = \{0.095, 0.11, 0.15\}~Mbps.  Given the packet payload size and data rate, the maximum throughput that can be achieved is about 0.124~Mbps for LT stations. As such, $T = 0.15$~Mbps represents a near-saturated state, $T = 0.11$~Mbps represents a medium traffic load and $T = 0.095$~Mbps results in low traffic load. %Each station randomly chooses its transmission transmission interval, and the ratio between maximal and minimal transmission interval is up to 20 \cite{Sensor2017}. 
%As \gls{etaroa} was already shown to outperform \gls{edca} for homogeneous stations, we use \gls{etaroa} as a benchmark in this section~\cite{Sensys2017}.


In this section, we evaluate the performance of \gls{moroa} for homogeneous stations (in terms of \gls{mcs} and packet size) with the multi-objective function weight $\alpha$  (cf. Eq.~\ref{eq:objective}) equals $0$ (i.e., pure energy optimization) and $1$ (i.e., pure throughput optimization) for a variety of traffic loads and station counts. Two different total traffic loads are simulated, i.e., $T$ = \{0.11, 0.15\}~Mbps \textcolor{red}{ for the LT scenario, and $T$ = \{0.75, 1.10\}~Mbps for the HT scenario}.  Given the packet payload size and \gls{mcs}, the maximum throughput that can be achieved is about 0.124~Mbps \textcolor{red}{and 0.90~Mbps for LT and HT stations respectively}. As such, $T = 0.15$~Mbps \textcolor{red}{and $T = 1.10$~Mbps} represent a saturated state, $T = 0.11$~Mbps \textcolor{red}{and $T = 0.75$~Mbps} represents a medium to high traffic load. Each station is assigned a random transmission interval, with the ratio between maximum and minimum transmission interval at most 20. 
As \gls{etaroa} was already shown to outperform \gls{edca} and other \gls{raw} optimization algorithms for homogeneous scenarios, we use \gls{etaroa} as a benchmark in this section~\cite{Sensys2017}.

%Figure~\ref{fig:sumo-home} depicts performance (i.e., throughput and latency) of \gls{moroa} and \gls{etaroa}. It clearly shows that, compared to \gls{etaroa}, \gls{moroa} improves the throughput under near saturated traffic conditions for any number of stations. At low traffic loads, \gls{moroa} and \gls{etaroa} have the same throughput for any number of stations. Under medium traffic conditions, \gls{moroa} also improves performance for a low number of stations. More importantly, better latency performance is achieved in denser networks. For a low traffic load (i.e., $0.095$~Mbps), \gls{moroa} and \gls{etaroa} have  nearly the same latency when there are less than 1024 stations. However, for 2048 stations, latency of \gls{moroa} is about $32.5$\% lower. For higher traffic loads, \gls{moroa} has better latency even in less dense networks (i.e., starting from 512 and 128 stations for traffic load $0.11$ and $0.15$~Mbps respectively). The above results reveal that, by using surrogate modeling, stations are assigned to RAW groups in a more optimal way. This is due to the fact that \gls{etaroa} derives its optimal \gls{raw} parameters from a model under saturated state as an approximation, rather than optimizing them for different traffic conditions as the surrogate model does. This results in better overall performance in terms of throughput and latency.

Figure~\ref{fig:sumo-home} depicts performance in terms of throughput and energy consumption of \gls{moroa} and \gls{etaroa}. It clearly shows that, compared to \gls{etaroa}, \gls{moroa} can tune the values of $\alpha$ to optimize the required performance objective. For $\alpha = 1$, \gls{moroa} improves the throughput  up to 5.3\% under saturated traffic conditions for any number of stations, \textcolor{red}{for both LT and HT stations}. Under medium traffic conditions, \gls{moroa} also improves performance. While \gls{moroa} consumes more energy than \gls{etaroa} does for both saturated and medium traffic conditions, up to 30\%. 
%Moreover, \gls{etaroa} has the similar power consumption for both traffic load, while the power consumption of \gls{moroa} increases when the network traffic load changes. 
Conversely, when $\alpha$ is $0$, \gls{moroa} consumes up to 22.6\% less energy at the cost of 4.4\% throughput decrease \textcolor{red}{for LT stations, and up to 32.4\% less energy at the cost of 3.26\% throughput decrease for HT stations}. This is due to the fact that \gls{etaroa} derives its optimal \gls{raw} parameters from a model of throughput under saturated state as an approximation, rather than optimizing them for different traffic conditions and objectives as the surrogate model does. This results in better overall performance in terms of throughput and energy efficiency. Figure \ref{fig:homo-alpha} zooms into \gls{moroa} performance with different $\alpha$ for 1024 \textcolor{red}{LT and 1024 HT stations,  with traffic load 0.15 Mpbs and 1.10 Mbps respectively}. It clearly shows, Compared to \gls{etaroa}, that by adjusting the value of $\alpha$, \gls{moroa} can either obtain higher throughput or lower power consumption. Moreover, the energy consumption is significantly affected by $\alpha$, up to 22.6\% \textcolor{red}{ and 21.2\%} can be saved \textcolor{red}{for LT and HT stations respectively} . While $\alpha$ has a small impact on throughput, only 4.5\% \textcolor{red}{and 4.0\%} throughput can be improved \textcolor{red}{for LT and HT stations respectively}. \textcolor{black}{
The above results reveal that, by using surrogate modeling, stations can be assigned to RAW groups in a more optimal way for the desired performance objective by adapting the $\alpha$ value. Moreover, these results show that by employing multi-objective optimization, a small reduction in throughput can lead to a significant improvement in terms of energy efficiency.}
%This is due to the fact that \gls{etaroa} derives its optimal \gls{raw} parameters from a model under saturated state as an approximation, rather than optimizing them for different traffic conditions as the surrogate model does. This results in better overall performance in terms of throughput and latency.



\subsection{Heterogeneous stations}
%Compare original E-TAROA (one station per slot) and revised E-TAROA (using SUMO modeling) in terms of throughput, latency and energy consumption. 
% 


%\begin{figure*}[t]
%    \subfloat[Throughput \label{fig:throughput-alpha}]{\includegraphics[width=0.5\textwidth]{image/algorithm/throughput-alpha}}
%    \subfloat[Fairness \label{fig:fairness-alpha-percent}]{\includegraphics[width=0.5\textwidth]{image/algorithm/fariness-alpha-percentage}}
%  \caption{Performance of \gls{moroa} with different $\alpha$ with heterogeneous stations. \label{fig:sumo-alpha}}
%\end{figure*}

% \begin{figure}[t]
%     \centering
%     \subfloat[Throughput \label{fig:throughput-alpha-0}]{\includegraphics[width=0.35\textwidth]{image/algorithm/final-throughput-merge-alpha-0}}\\
%     \subfloat[Energy consumption \label{fig:delay-alpha-0}]{\includegraphics[width=0.35\textwidth]{image/algorithm/final-fariness-merge-alpha-0percentage}}
%   \caption{\textcolor{black}{Performance comparison between \gls{moroa} and \gls{edca}, for a heterogeneous network.} \label{fig:sumo-throughput-csma}}
% \end{figure}

\begin{figure*}[t]
    \centering
    \subfloat[Throughput, LT \label{fig:throughput-alpha-0}]{\includegraphics[width=0.45\columnwidth]{image/algorithm/Final-throughput-alpha-LT}}
    \subfloat[Throughput, HT \label{fig:delay-alpha-0}]{\includegraphics[width=0.45\columnwidth]{image/algorithm/Final-throughput-alpha-HT}} \\
     \subfloat[Energy consumption, LT \label{fig:throughput-alpha-0}]{\includegraphics[width=0.45\columnwidth]{image/algorithm/Final-energy-alpha-LT}}
    \subfloat[Energy consumption, HT \label{fig:delay-alpha-0}]{\includegraphics[width=0.45\columnwidth]{image/algorithm/Final-energy-alpha-HT}}
  \caption{\textcolor{red}{Performance comparison between \gls{moroa} using different $\alpha$ and \gls{edca}, for a heterogeneous network} \label{fig:sumo-throughput-csma}}
\end{figure*}




In this section, we study the performance of \gls{moroa} in a network with heterogeneous stations. Moreover, we evaluate the effects of the $\alpha$ weight parameter on throughput and \textcolor{black}{energy consumption of both LT and HT stations}, for a variety of network densities. As \gls{etaroa} only supports stations with same \gls{mcs} and packet size, the results of \gls{moroa} are compared to \gls{edca} in this case. 

%Figure \ref{fig:throughput-alpha} shows that by increasing the value of $\alpha$, higher throughput can be achieved. Between $\alpha=0.25$ and $\alpha=0.5$, throughput increases $22$\%, $36$\%, $44$\% for $1200$, $1400$ and $1600$ stations respectively. While for $\alpha$ less than $0.25$ or larger than $0.5$, the throughput does not significantly change. For the small size networks, it shows there is no need to tune $\alpha$. Figure~\ref{fig:fairness-alpha-percent} illustrates the fairness with different $\alpha$, depicting the same conclusion.  With smaller values of $\alpha$, more fairness is achieved in terms of relative transmitted number of packets between LT and HT stations. Increasing $\alpha$ results in more focus on throughput, resulting in more airtime being provided to the HT stations. %The interesting finding from figure \ref{fig:sumo-alpha} is, although $\alpha$ is defined between 0 and 1, the actual range that affects the results is much smaller. More investigation are needed about tunning the $\alpha$, in order to achieve the required performance, it is considered as  our next step. 

%The comparison between \gls{moroa} and \gls{edca} is depicted in figure \ref{fig:sumo-throughput-csma}. With a small number of stations, \gls{moroa} and \gls{edca} have the same throughput. With more than 600 stations, the throughput of \gls{edca} stagnates to about 0.25~Mbps as the network becomes saturated. However, \gls{moroa} reduces the number of collisions, resulting in better scaling behavior and a higher saturation limit. As a result, for 1800 stations throughput of \gls{moroa} is about 44\% higher for $\alpha=0$ and  $\alpha=0.25$. If more focus is put on the throughput objective, with $\alpha=0.75$, \gls{moroa} more than doubles throughput compared to \gls{edca} in highly dense networks (i.e., 1800 stations), achieving a throughput increase of 108\%. As expected, \gls{edca} achieves high fairness among HT and LT stations due to its random access mechanism. For $\alpha \leq 0.25$ \gls{moroa} achieves a similar degree of fairness. However, for $\alpha = 0.75$ fairness of \gls{moroa} is significantly reduced, as it prioritizes HT stations to maximize overall throughput.

The comparison between \gls{moroa} and \gls{edca} is depicted in Figure~\ref{fig:sumo-throughput-csma}. With a small number of stations (i.e., 400 stations), \gls{moroa} and \gls{edca} have the same throughput. \textcolor{red}{For 600 stations, \gls{edca} still remains similar throughput as \gls{moroa} with $\alpha$ larger than 0.5.} While with more than 1000 stations, the throughput of \gls{edca} stagnates to about 0.047~Mbps and 0.21~Mbps for LT and HT stations respectively, as the network becomes saturated. However, for $\alpha = 1$ and $\alpha = 0.75$, \gls{moroa} reduces the number of collisions and results in better scaling behavior and a higher saturation limit. For example, with 1600 stations and $\alpha=0.75$, throughput of \gls{moroa} is about 38\% and 20\% higher for LT and HT stations, respectively. If more focus is put on the throughput objective, with $\alpha=1$, \gls{moroa} achieves about 65\% and 47\% higher throughput for LT and HT stations, respectively. In contrast, when $\alpha = 0$, \gls{moroa} obtains 31\% and 64\% less throughout for  LT and HT stations than \gls{edca} does.
In terms of energy consumption, \gls{moroa} always significantly outperforms \gls{edca}. For  1600 stations with $\alpha = 0$,  \gls{moroa} saves about 96\% and 91\% energy for LT and HT stations, respectively. 
%It should be noted that $\alpha = 0$ and $\alpha = 0.5$ result in the quite similar performance, it is because the range of objective $Q_t$ and $Q_e$ are different. As such, the $\alpha$ can be actually tuned between $0.5$ and $0.1$ to achieve the desired performance.

\begin{figure}[t]
  \centering
  \includegraphics[width=0.75\columnwidth]{image/algorithm/Final-collision-alpha-zoom-submission}
  \caption{\textcolor{red}{Collision comparison between MoROA using different $\alpha$ and EDCA/DCF, for a heterogeneous network.} \label{fig:collison}}
\end{figure}

\textcolor{red}{ The number of packets that collide with each other were counted during simulation, the results are depicted in Figure \ref{fig:collison}. It  clearly shows that, compare to \gls{edca}, \gls{moroa} significantly reduce the number of collisions. For 400 and 600 stations, the number of collision is 52845 and 60344 for \gls{edca}. As the number of station increases, the number of collision goes up dramatically, around 1122775 for 1600 stations. In contrast, the number of collision \gls{moroa} is at most 29142, 97.4\% less than \gls{edca}. As collision may cause the packets to be dropped, \gls{moroa} obtains higher performance (e.g., throughput, energy) than \gls{edca}. It should be noted that \gls{moroa} with $alpha = 0$  have less collision than \gls{edca}, but lower throughput than \gls{edca}. This is due to the fact \gls{moroa} with $alpha = 0$ tries to avoid the collision as much as possible, in order to minimize the energy consumption. As such, the airtime is not fully utilized, leading to lower throughput.}




%\begin{figure}[t]
%  \centering
%  \includegraphics[width=\columnwidth]{image/algorithm/Final-throughput-alpha-0}
%  \caption{Performance comparison between \gls{moroa}($\alpha=0.75$, $\alpha=0.25$, $\alpha=0.00$), and EDCA/DCF for heterogeneous network for LT and HT stations respectively. \label{fig:sumo-iteraton}}
%\end{figure}



% \begin{figure*}[t]
%     \subfloat[Ratio 25\%:75\%, throughput \label{fig:traffic-pattern-throughput}]{\includegraphics[width=0.5\textwidth]{image/algorithm/throughput-static-taroa-nc-aidre-025}}
%     \subfloat[Ratio 25\%:75\%, latency \label{fig:traffic-pattern-throughput}]{\includegraphics[width=0.5\textwidth]{image/algorithm/delay-static-taroa-nc-aidre-025}}
%  % \caption{Performance comparison between SUMO-E-TAROA and EDCA/DCF for mixed scenario with 25\% LT stations and 75\% stations. \label{fig:sumo-throughput}}
% %\end{figure*}

% %\begin{figure*}[t]
%     \subfloat[Ratio 50\%:50\%, throughput \label{fig:traffic-pattern-throughput}]{\includegraphics[width=0.5\textwidth]{image/algorithm/throughput-static-taroa-nc-aidre-050}}
%     \subfloat[Ratio 50\%:50\%, latency \label{fig:traffic-pattern-throughput}]{\includegraphics[width=0.5\textwidth]{image/algorithm/delay-static-taroa-nc-aidre-050}}
% %  \caption{Performance comparison between SUMO-E-TAROA and EDCA/DCF for mixed scenario with 50\% LT stations and 50\% stations. \label{fig:sumo-throughput}}
% %\end{figure*}






%\subsection{dynamic traffic pattern (optional)}


